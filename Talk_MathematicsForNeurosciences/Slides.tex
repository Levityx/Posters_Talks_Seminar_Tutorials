\documentclass{beamer}
\mode<presentation>
\usepackage[french,british]{babel}
\usepackage[T1]{fontenc}

\setbeamertemplate{footline}[page number]{}  
\setbeamertemplate{navigation symbols}{}  

\title[Maths pour Neuro]{Math�matiques pour les neurosciences\\
\small{une petite pr�sentation}}
\author[Levity]{
\raisebox{0.4cm}{Al Levity}\\
\includegraphics[height=1.1cm]{Logos/RxPz.jpg}\\[1.3cm]
\includegraphics[height=1.0cm]{Logos/ihr2.png}\hspace{0.8cm}
\includegraphics[height=1.5cm]{Logos/Logo.jpg}  \hspace{0.7cm}
\includegraphics[height=1.0cm]{Logos/notts.jpg}\\
\textcolor{gray}{allevity@ihr.mrc.ac.uk}
}
\date{}

\usetheme{Singapore}




%%%%
\usepackage{animate}
\usepackage{multimedia}
\usepackage{graphics}
\usepackage{tikz}
\usepackage{pgfplots}

\usepackage{amsmath}
\usepackage{amsfonts}
\definecolor{shadecolor}{RGB}{1., 1., 0.} 

\newcommand{\Vth}{V_\text{th}}
\newcommand{\dx}{\,\text{d}x}
\newcommand{\dy}{\,\text{d}y}
\newcommand{\dt}{\,\text{d}t}
\newcommand{\ds}{\,\text{d}s}
\newcommand{\dB}{\,\text{d}B}
\newcommand{\Lgen}{\mathcal{L}}
\newcommand{\Rmq}{\noindent \textbf{Remark:} }
\newcommand{\ToDo}[1]{\noindent\textcolor{red}{\textbf{#1}}}
\newcommand{\Thm}[2]{\noindent\textbf{Theorem #1:} \textit{#2}\\}
\newcommand{\Thmnospace}[2]{\noindent\textbf{Theorem #1:} \textit{#2}}

\definecolor{colorcoeff}{rgb}{1,0,1}
\definecolor{colorcoeff2}{rgb}{1,0,1}
\definecolor{colorparam}{rgb}{1,0,0}

\newcommand{\demi}{\frac{1}{2}}
\newcommand{\Demi}{\dfrac{1}{2}}

\DeclareMathOperator*{\E}{\mathbb{E}}
\DeclareMathOperator*{\R}{\mathbb{R}}
\DeclareMathOperator*{\N}{\mathbb{N}}
\DeclareMathOperator*{\Normal}{\mathcal{N}}

\newlength\figureheight 
\newlength\figurewidth 

%%%%

\begin{document}

\begin{frame} \titlepage \end{frame}

\section[Intro.]{Introduction}   
 
\setbeamercovered{transparent}

\subsection{Mod�liser}

\begin{frame} 
\begin{itemize}
\item Santiago Ram\'on y Cajal, fin du XIX�me si�cle, \\
p�re des neurosciences.\\
\pause
\item Hodgkin-Huxley, 1952\\
analogie entre un neurone et un syst�me �lectrique,\\ 
\pause
\item Aujourd'hui, on essaie de d�chiffrer le code neuronal.\\[5mm]
\pause
 \begin{center}Quels r�les ont les math�matiques l�-dedans ?\hspace{3mm}\\[4mm]
 \visible<5>{\textbf{Mod�liser et analyser.}}
\end{center}
 \end{itemize}
\end{frame}

%Pr�sentation biologique et fonctionnelle du cerveau, de macro � micro
%Pr�sentation de divers outils math�matiques
%Mod�lisation, de micro � macro

\begin{frame}
\frametitle{Mod�liser}
\vspace{3mm}What I cannot create I do not understand.\\[3mm]
\begin{center}\includegraphics[width = 8cm]{feynman.jpg}\end{center}
\hfill Richard Feynman (1918-1988) 
\end{frame}

\subsection[�chelles]{Les �chelles}

\begin{frame}
\frametitle{Les �chelles (scaling)}
Exemple physique : du macroscopique au microscopique.\\

\begin{tikzpicture}
 \visible<2->{ \node (img1) {\includegraphics[width=4cm]{Intro/universe.jpg}};}
  %\pause
 \visible<3->{ \node (img2) at (img1.south east) [xshift=-3mm, yshift=1cm] {\includegraphics[width=5cm]{Intro/earth.jpg}};}
  %\pause
  \visible<4->{ \node (img3) at (img2.south east) [xshift=-3mm,yshift=0.7cm] {\includegraphics[width=4cm]{Intro/human.jpg}};}
  %\pause
  \visible<5->{ \node (img4) at (img3.south east) [xshift=-3mm,yshift=0.3cm] {\includegraphics[width=3.5cm]{Intro/Cell.jpg}};}
  %\pause
  \visible<6->{ \node (img5) at (img4.south east) [xshift=-3mm,yshift=0.9cm] {\includegraphics[width=3.5cm]{Intro/water.jpg}};}
 
  \end{tikzpicture}

La physique th�orique cherche encore une th�orie unificatrice.\\
%http://htwins.net/scale2/\\
%http://micro.magnet.fsu.edu/primer/java/scienceopticsu/powersof10/index.html
\end{frame}

\begin{frame}
\frametitle{Les �chelles (scaling)}
Exemple Math�matique : un monde sans �chelles.\\
Les fractales (auto-similaires) \\[3mm]
\only<1>{\mbox{\phantom{\includegraphics[width=3.5cm]{Droste/Sierpinski.png}}}}%
\includegraphics<2->[width=3.5cm]{Droste/Sierpinski.png}\hspace{1mm}%
\only<-2>{\mbox{\phantom{\includegraphics[width=3.8cm]{Droste/escher.jpg}}}}%
\includegraphics<3->[width=3.8cm]{Droste/escher.jpg}\hspace{1mm}%
\only<-3>{\mbox{\phantom{\includegraphics[width=3cm]{Droste/00.jpg}}}}%
\includegraphics<4->[width=3cm]{Droste/00.jpg}\hspace{1mm}%
\only<-4>{\mbox{\phantom{(Escher, Droste effect, Sierpinski)}}}%
\only<5->{(Sierpinski, \hspace{1.7cm}Escher, \hspace{2.6cm}Droste effect)}
\end{frame}

\begin{frame}
\frametitle{Les �chelles (scaling)}
Exemple Math�matique : un monde d'�chelles.\\
Les fractales (non auto-similaires)\\[3mm]
\begin{center}
\only<1>{\mbox{\phantom{\includegraphics[width=9.5cm]{Fractals/Mandel.jpg}}}}%
\includegraphics<2->[width=9.cm]{Fractals/Mandel.jpg}
\end{center}
\only<-2>{\mbox{\phantom{http://www.htwins.net/mandyzoom/http://www.youtube.com/watch?v=tzNjmSGVs6o}}}
\only<3->{http://www.htwins.net/mandyzoom/\\http://www.youtube.com/watch?v=tzNjmSGVs6o}
\end{frame}

\begin{frame}
\frametitle{Les �chelles (scaling)}
Exemple Neuro : du macroscopique au microscopique.\\
\begin{tikzpicture}
 \visible<1->{ \node (img1) {\includegraphics[width=4.5cm]{BrainScales/1_brain.png}};}
  \pause
 \visible<2->{ \node (img2) at (img1.south east) [yshift=0.5cm] {\includegraphics[width=4.8cm]{BrainScales/2_brain.jpg}};}
  \pause
  \visible<3->{ \node (img3) at (img2.south east) [yshift=0.5cm] {\includegraphics[width=4cm]{BrainScales/3_vibrissal-cortex-rat.jpg}};}
  \pause
  \visible<1->{ \node (img4) at (img3.south east) [yshift=0.5cm] {\includegraphics[width=5cm]{BrainScales/4_neurons.png}};}
  \pause
  \visible<5->{ \node (img5) at (img4.west) [xshift=-2cm, yshift=-0.8cm] {\includegraphics[width=3cm]{BrainScales/6_Ionic.jpg}};}
 % \pause
   \visible<4->{\node (img6) at (img4.north) [xshift=0.4cm, yshift=1.5cm] {\includegraphics[width=3cm]{BrainScales/5_Synapse.png}};}
  \end{tikzpicture}
  \end{frame}

\subsection{Technologie}


\begin{frame}
\frametitle{D�couvertes et technologie}
\begin{tabular}{ c c c }
  Santiago Ram\'on y Cajal, 1899 && Lynne Quarmby, 2011 \\[3mm]
  \includegraphics[height=4cm]{DecouvertesTechn/1899_PurkinjeCell.jpg} && \includegraphics[height=4cm]{DecouvertesTechn/2011_cell.jpg} \\
\end{tabular}
\end{frame}

\section[Neur.]{Neurones}      
 \begin{frame}\begin{center}
	$\mathcal{NEURONES}$
	\end{center}
\end{frame}



\subsection{}%%%
\begin{frame}
\frametitle{Quelques nombres}
\begin{itemize}
\item $\approx$100.000.000.000 neurones chez l'Homme %\only<2>{\\[2mm]}
\only<1>{\mbox{\phantom{\hspace{3mm}\includegraphics[height=6.9cm]{Neurons/Brains.jpg}}}}
\only<2>{\begin{center}\hspace{3mm}\\[-2mm]\includegraphics[height=6.9cm]{Neurons/Brains.jpg}\end{center}}
\only<3->{\item Jusqu'� 10.000 connections par neurone}
\only<3>{\mbox{\phantom{\hspace{1mm}\includegraphics[height=6.5cm]{Neurons/neuronGrowthDeath.jpg}}}}
%http://www.theemotionmachine.com/mindfulness-and-neuroplasticity
\only<4>{\includegraphics[height=6.5cm]{Neurons/neuronGrowthDeath.jpg}}
\only<5->{\item Plusieurs types de neurones}
\only<5>{\mbox{\phantom{\hspace{1mm}\includegraphics[height=6cm]{Neurons/neurontypes.png}}}}
%http://www.mind.ilstu.edu/curriculum/neurons_intro/imgs/neuron_types.gif
\only<6>{\includegraphics[height=6cm]{Neurons/neurontypes.png}}
\only<7->{\item Diff�rentes �chelles possibles\\}
\only<7>{\mbox{\phantom{\hspace{1mm}\includegraphics[height=5.5cm]{Neurons/lengths.jpg}}}}
%http://thebrain.mcgill.ca/flash/d/d_06/d_06_cl/d_06_cl_mou/d_06_cl_mou_1a.jpg
\only<8>{\begin{center}\includegraphics[height=5.5cm]{Neurons/lengths.jpg}\end{center}}
\end{itemize}
\end{frame}

\subsection{R�seaux}%%%
\begin{frame}
\frametitle{Connectome}
D'o� des connections difficiles � analyser.\\
\begin{center}\includegraphics[height=5.5cm]{Neurons/neural-network-model.jpg}\end{center}
%http://4.bp.blogspot.com/-7roS9m_NcGk/Ui7yhdCTEYI/AAAAAAAAASU/iAK3De3Yahg/s1600/neural-network-model.jpg
\end{frame}

\subsection{Activit� �lectrique}%%%
\begin{frame}
\frametitle{Neurones : deux r�gimes �lectriques}
\begin{center}
Information neuronale �lectrique $\implies$ �lectrodes\\[3mm]
\only<1>{\includegraphics[height=5.5cm]{Neurons/electrode.jpg}}
%http://hirnforschung.kyb.mpg.de/uploads/pics/EN_M1_clip_image002_03.jpg
\only<2->{\includegraphics[height=5.5cm]{Neurons/electrode2.jpg}}
\end{center}
\end{frame}

%Leur demander comment s'occuper de cette somme. 
% Ils donneront des r�ponses physiques (pousser les autres), et leur dire qu'il y a des m�thodes math�matiques (ICA, PCA, ...)

\begin{frame}
\frametitle{Neurones : deux r�gimes �lectriques}
\hspace{-10mm}
\begin{center}
\begin{minipage}{5mm}
\tiny{Voltage (mV)}\\
\begin{tikzpicture}[scale=1]
\draw [->] (-5,-5) -- (-5,0);
\end{tikzpicture}
\end{minipage}
\hspace{-2mm}
\begin{minipage}{10cm}
\only<1>{\includegraphics[width=11cm]{Neurons/Data0_0.png}}
\only<2>{\includegraphics[width=11cm]{Neurons/Data0_1.png}}
\only<3, 4->{\includegraphics[width=11cm]{Neurons/Data0.png}\\}
\only<4>{\includegraphics[width=11cm]{Neurons/Data52.png}}
\only<1, 2, 3>{\mbox{\phantom{\includegraphics[width=11cm]{Neurons/Data52.png}}}}
\end{minipage}

\vspace{-2.mm}
\begin{tikzpicture}[scale=1]
\draw [->] (-2.3,0) -- (7,0);
\end{tikzpicture}
\tiny{Time (ms)}
\end{center}
\end{frame}


\begin{frame}
\frametitle{Dynamique des potentiels d'action}
\begin{center}
Les spikes se ressemblent \\[2mm]
\includegraphics[width=10.5cm]{Neurons/15000pointsofzelle17septthespikes2}
\end{center}
\end{frame}



\section[Math]{Outils math�matiques 1}  
 
\def\FunctionF(#1){(#1)^3- 3*(#1)}%
\def\FunctionFPrime(#1){3*(#1)^2- 3}%

\begin{frame}
\begin{center}
	$\mathcal{OUTILS\,\,\, MATHEMATIQUES}$
\end{center}
\end{frame}

\subsection{D�rivation}
\begin{frame}
\frametitle{D�rivation}
\begin{tabular}{c  c}
Graphe de la fonction & \only<2,3,4,5>{D�riv�e de $f$:}\\[2mm]
$f:
\left\{
	\begin{array}{ll}
		\R \to \R\\
		x\mapsto x^3-3x
	\end{array}
\right.$ & \visible<2,3,4,5>{$f':
\left\{
	\begin{array}{ll}
		\R \to \R\\
		x\mapsto\only<2,3,4>{?}\visible<5>{3x^2-3}\only<5>{\visible<4>{?}}
	\end{array}
\right.$}\\[4mm]
\begin{minipage}{5cm}
\begin{tikzpicture}[scale=0.5]
\begin{axis}[
        axis y line=center,
        axis x line=middle, 
        axis on top=true,
        xmin=-5.5,
        xmax=5.5,
        ymin=-45,
        ymax=45,
        height=12.0cm,
        width=12.0cm,
        grid,
        xtick={-5,...,5},
        ytick={-40,-32,...,40},
    ]
    \addplot [domain=-5:5, samples=50, mark=none, ultra thick, blue] {\FunctionF(x)};
    \node [left, blue] at (axis cs: 3.6,42) {$x^3-3x$};
\end{axis}
\end{tikzpicture}
\end{minipage}&%
\begin{minipage}{5.5cm}
\begin{center}
\only<1,2,3,4>{
\visible<3,4>{\textit{D�finition :} pour $x\in\R,$
$$f'(x) := \lim_{h\to0}\frac{f(x+h)-f(x)}{h}$$}
\visible<4>{\textit{Propri�t� :} soient $f$ et $g$ deux fonctions d�rivables, $a, b\in\R, n\in\N^*$. Alors\\ [1mm]
$(a f + b g)' = a f' + b g'$\\[1mm]
$(fg)' = f'*g + f*g'$\\[1mm]
$(f^n)' = n(f')f^{n-1}$}
}
\only<5>{{
\begin{tikzpicture}[scale=0.5]
\begin{axis}[
        axis y line=center,
        axis x line=middle, 
        axis on top=true,
        xmin=-5.5,
        xmax=5.5,
        ymin=-45,
        ymax=45,
        height=12.0cm,
        width=12.0cm,
        grid,
        xtick={-5,...,5},
        ytick={-40,-32,...,40},
    ]
    \addplot [domain=-5:5, samples=50, mark=none, ultra thick, red] {\FunctionFPrime(x)};
    \node [left, red] at (axis cs: 3.6,42) {$3x^2-3$};
\end{axis}
\end{tikzpicture}}}
\end{center}
\end{minipage}
\end{tabular}
\end{frame}

\subsection{Equations diff�rentielles}
\begin{frame}
\frametitle{Equations diff�rentielles}
\begin{center}
\'Equation dont l'inconnue est une fonction $f$,  entre $f$ et ses d�riv�es.\\
\visible<2,3,4>{Exemples :}
\visible<2>{$f'(x)=f(x)$}
\visible<3>{$f' = f^2-3f+2$}
\visible<4>{$f'(x) +f''(x) = \frac{f^2(x) + 1}{1+x^2} $}
\visible<5,6,7,8>{Existence et unicit� d'une solution ?}\\
\visible<6,7,8>{Pr�ciser des conditions aux limites.}\\
\visible<7,8>{Trouver le domaine.}\\[6mm]
\visible<8>{De nombreux th�or�mes (Lipshitz), de nombreuses conjectures.}
\end{center}
\end{frame}

\begin{frame}
\frametitle{Equations diff�rentielles}
\visible<1>{Comment tracer les solutions ?}\\[3mm]
\visible<2,3,4>{R�solution \underline{graphique} de 
$\left\{
	\begin{array}{ll}
		f' = f \\
		f(0) = 1.
	\end{array}
\right.$}
\begin{center}
\begin{tikzpicture}[scale=0.5]
\begin{axis}[
        axis y line=center,
        axis x line=middle, 
        axis on top=true,
        xmin=-5,
        xmax=5,
        ymin=-1,
        ymax=10,
        height=12.0cm,
        width=12.0cm,
        grid,
        xtick={-5,...,5},
        ytick={-1,0,...,10},
    ]

\only<3>{\addplot[ samples=41, mark=*, blue]{exp(x)};}
\only<4>{\addplot[ samples=61, mark=none, blue]{exp(x)};}
\end{axis}
\end{tikzpicture}
\end{center}
\end{frame}

\begin{frame}	
\frametitle{Equations diff�rentielles}
\begin{center}
\includegraphics[width=11cm]{EquaDiff.jpg}
\end{center}
\end{frame}

\subsection{Al�atoire}
\begin{frame}
\frametitle{Mouvement brownien}
\begin{center}
\begin{tabular}{c c c}
 1827&1905&1921\\[1mm]
 \hline\\ [-1.8ex]
 \visible<2,3,4>{Biologie}&\visible<3,4>{Physique}&\visible<4>{Math�matiques}\\[1mm]
 \hline \\ [-1.5ex]
 \begin{minipage}{3.0cm}
  \visible<2,3,4>{\footnotesize{\textbf{Robert Brown} remarque au microscope que le mouvement de grains de pollen sur l'eau ne suit pas de logique �vidente.}}
 \end{minipage}
 &
  \begin{minipage}{3.0cm}
  \visible<3,4>{\footnotesize{\textbf{Albert Einstein} publie une th�orie quantitative du mouvement brownien.}}\end{minipage}
  &
   \begin{minipage}{3.1cm}
 \visible<4>{  \footnotesize{\textbf{Norbert Wiener} propose un cadre formel au mouvement brownien\,: le processus de Wiener.}}\end{minipage}
\end{tabular}
\end{center}
\end{frame}

\begin{frame}
\frametitle{Mouvement brownien}
\begin{center}
\includegraphics[width=11cm]{MvtBr.jpg}
\end{center}
\end{frame}

\begin{frame}
\frametitle{Mouvement brownien}
Permet une th�orie du calcul stochastique \\(des �quations diff�rentielles qui contiennent de l'al�a)\\
\visible<1,2,3,4>{$$f'(t) = -f(t) \,\textcolor{gray}{\to}\, \visible<2,3,4>{\frac{\text{d}f}{\dt}(t) = -f(t)}\,\textcolor{gray}{\to}\,  {\footnotesize\visible<3,4>{\frac{\text{d}f}{\dt} = -f} \,\textcolor{gray}{\to}\, \visible<4>{\text{d}f = -f\, \dt}} $$}
\visible<5,6>{
\begin{tabular}{l l}
Calcul classique : &$\text{d}f = -f\, \dt$\\
\visible<6>{Calcul stochastique : }&\visible<6>{$\text{d}f = -f\, \dt+\text{d}B_t$}
\end{tabular}
 }
\end{frame}

\section[Mod�l.]{Mod�lisation} 
 
\begin{frame}
\begin{center}
	$\mathcal{MODELISATION}$
\end{center}
\end{frame}

\subsection{Cas d'�tude : Hodgkin-Huxley}
\begin{frame}
\frametitle{Hodgkin-Huxley model (HH)}
Mod�liser le voltage d'un neurone en mod�lisant les m�canismes biologiques qui le font varier.
$$V'(t) = \only<1-7>{\quad ?}\visible<8>{f(n_{K^+}, n_{Na^+}, V(t))}
 \only<8>{\visible<7>{\quad ?}}$$
\visible<2,3,4,5,6,7,8>{Qu'est-ce qui d�termine le potentiel �lectronique dans un neurone ?\\}
\visible<3,4,5,6,7,8>{Les ions :}
\begin{itemize}
\visible<4,5,6,7,8>{\item Potassium $K^+$}
\visible<5,6,7,8>{\item Sodium $Na^+$}
\visible<6,7,8>{\item ...}
\end{itemize}
\visible<7,8>{On va donc construire notre mod�le en choisissant comment nos param�tres influent sur  la variation du voltage.}
\end{frame}




\begin{frame}
\frametitle{Dynamique des potentiels d'action}
\begin{center}
	\includegraphics[height=7.5cm]{Neurons/AP.jpg}
\end{center}
\end{frame}



\begin{frame}
\frametitle{Potentiels d'action : jeu de barycentre}
\begin{center}
	\includegraphics[height=6.5cm]{HHModel/AP.png}
\end{center}
\end{frame}



\subsection{HH model}
\begin{frame}
\frametitle{HH model : construction}
\only<1>{Comme ce sont les \textcolor{red}{propri�t�s �lectriques} du neurone qui cr�ent les potentiels d'action, on va construire notre mod�le par analogie avec un  \textcolor{red}{syst�me �lectrique}.
}

\begin{center}
\only<2>{\includegraphics[height=6.5cm]{HHModel/circuit0.png}\\[2mm]
\footnotesize{En 1963, ils recevaient le prix Nobel de physiologie ou m�decine pour ce travail.}
}
\only<3>{\includegraphics[height=3.5cm]{HHModel/circuit1.png}}
\only<4>{\includegraphics[height=3.5cm]{HHModel/circuit2.png}}
\only<5>{\includegraphics[height=6.5cm]{HHModel/circuit3.png}}
\end{center}

\only<6-9>{
\footnotesize{Dans un circuit en parall�le, les courants s'ajoutent :}
$$I_{total} = I_C + I_{K}+I_{Na}+I_{l}$$
\only<6>{\begin{itemize}
\item \footnotesize{$I_C$ est le courant passant � travers la bicouche lipidique,}
\item \footnotesize{$I_K$, $I_{Na}$ sont les courants � travers les canaux ioniques $K$ et $Na$,}
\item \footnotesize{$I_l$ est un courant de fuite (leakage current), principalement via des fluctuations en ions chlorure ($Cl^-$).}
\end{itemize}}

\visible<7-9>{\footnotesize{On peut utiliser la loi d'Ohm (R R�sistance, g  Conductance) :}
$$I_i = \frac{V_i}{R_i} = g_i V_i $$}
\visible<8-9>{\footnotesize{Comme l'intensit� due � chaque type de courant est par rapport � une valeur "de repos"}
$$I_{K} = g_k(V-V_K)$$
$$I_{Na} = g_{Na}(V-V_{Na})$$
$$I_{l} = g_l(V-V_l)$$}
\visible<9>{\footnotesize{� part le courant $I_C$ qui est r�git par une autre loi}
$$I_C = C \frac{\text{d}V}{\text{d}t}$$
}
}

\only<10-11>{On se retrouve donc avec une formule plus int�ressante 
$$I=C\frac{\text{d}V}{\text{d}t} +  g_k(V-V_K) + g_{Na}(V-V_{Na}) +g_l(V-V_l) $$
\visible<11>{Qu'est-ce qui est \textcolor{blue}{constant}, qu'est-ce qui est \textcolor{red}{variable} ?}
\visible<13>{ Dans notre travail de mod�lisation, il nous reste �
\begin{itemize}
\item �valuer les constantes (les biologistes sont nos amis), 
\item mod�liser les variables $g_K, g_{Na}$.
\end{itemize}
}
}

\only<12-13>{
\visible<12>{On se retrouve donc avec une formule plus int�ressante }
$$\textcolor{red}{I}=
\textcolor{blue}{C}
\textcolor{red}{\frac{\text{d}V}{\text{d}t} }+ 
 \textcolor{red}{g_k}(\textcolor{red}{V}-\textcolor{blue}{V_K}) + 
 \textcolor{red}{g_{Na}}(\textcolor{red}{V}-\textcolor{blue}{V_{Na}}) +
 \textcolor{blue}{g_l}(\textcolor{red}{V}-\textcolor{blue}{V_l}) $$
\visible<12>{Qu'est-ce qui est \textcolor{blue}{constant}, qu'est-ce qui est \textcolor{red}{variable} ?\\}
\visible<13>{ Dans notre travail de mod�lisation, il nous reste �
\begin{itemize}
\item �valuer les constantes (les biologistes sont nos amis), 
\item mod�liser les variables $g_K, g_{Na}$.
\end{itemize}
}
}


\only<14-14>{Comme la conductance $g_K$ est maximale lorsque tous les canaux ioniques de potassium sont actifs (i.e. laissent passer des ions), on �crit tout d'abord
$$g_K = \overline{g_K} \;\times\;\textcolor{red}{?}$$
o� \textcolor{red}{?} est une variable entre 0 et 1 donnant le pourcentage de canaux actifs.
}

\only<15>{\begin{center}\includegraphics[height=4.5cm]{HHModel/activation.png}
\end{center}%\\[4mm]
Pour qu'un canal ionique soit actif, il faut 
\begin{itemize}
\item qu'il ne soit pas d�sactiv�,
\item qu'il ne soit pas inactiv�.
\end{itemize}
}

\only<16-18>{Des donn�es permettent l'estimation suivante des canaux ioniques : 
\begin{itemize}
\item $K^+$ \;\,: 4 portes pour la d�sactiver, 0 pour l'inactiver.
\item  $Na^+$ : 3 portes pour la d�sactiver, 1 pour l'inactiver.\\[3mm]
\end{itemize}

\visible<17-18>{Ainsi, si on note $n$ la probabilit� qu'une porte d�sactive un canal $K^+$, la probabilit� que le canal soit ouvert est $n^4$ (ind�pendance des �v�nements).\\[3mm]}

\visible<18>{Si on note  $m$ la probabilit� qu'une porte d�sactive un canal $Na^+$, $h$ la probabilit� qu'une porte inactive le canal, alors la probabilit� que le canal soit ouvert est $m^3h$.}
}
\only<19-22>{Notre mod�le devient donc
$$\textcolor{red}{I}=
\textcolor{blue}{C}
\textcolor{red}{\frac{\text{d}V}{\text{d}t} }+ 
 \textcolor{blue}{\overline{g_K}}\textcolor{red}{n^4}(\textcolor{red}{V}-\textcolor{blue}{V_K}) + 
\textcolor{blue}{\overline{g_{Na}}}\textcolor{red}{m^3h}(\textcolor{red}{V}-\textcolor{blue}{V_{Na}}) +
 \textcolor{blue}{\overline{g_l}}(\textcolor{red}{V}-\textcolor{blue}{V_l}) $$
et il nous reste � mod�liser la variation des variables $m, n, h$. \\[3mm]

\visible<20-22>{Pour certaines raisons, Hodgkin et Huxley ont cherch� � mod�liser sous la forme suivante : \\ }

\visible<21-22>{$$\begin{array}{c l}
\dot{\textcolor{colorparam}{n}} = \textcolor{colorcoeff}{\alpha_n(V)} \,\ (1 - \textcolor{colorparam}{n}) \; - \textcolor{colorcoeff}{\beta_n(V)}\; \; \textcolor{colorparam}{n}, \\[2mm]
\dot{\textcolor{colorparam}{m}} = \textcolor{colorcoeff}{\alpha_m(V)}\ (1 - \textcolor{colorparam}{m}) - \textcolor{colorcoeff}{\beta_m(V)}\ \textcolor{colorparam}{m}, \\[2mm]
\dot{\textcolor{colorparam}{h}} =  \textcolor{colorcoeff}{\alpha_h(V)} \; \;  (1 - \textcolor{colorparam}{h}) \;  - \textcolor{colorcoeff}{\beta_h(V)}\; \;  \textcolor{colorparam}{h}, \\[2mm]
\end{array}$$}

\visible<22>{\begin{center}JACKPOT\end{center}}
}
\only<23>{\raisebox{3cm}{
\begin{minipage}{5cm}
Les biologistes ont des techniques pour figer les canaux ioniques, permettant d'�valuer les fonctions $  \textcolor{colorcoeff}{\alpha_i(V)}  $ et $\textcolor{colorcoeff}{\beta_i(V)} $ une par une.\\[3mm]
Les �quations de ces fonctions sont ensuites ajust�es pour approcher la 'voltage-clamped' data.
\end{minipage}\hspace{0.5cm}
\begin{minipage}{5cm}
{\includegraphics[height=6.0cm]{HHModel/clamp2.png}}
\end{minipage}
}
}
\only<24-25>{
Les �quations propos�es en 1952 furent :\\[6mm]
\begin{center}
\begin{tabular}{l c l c l  c l }
$\textcolor{colorcoeff}{\alpha_n(V)} $&=& $0.01 \ \frac{10 - \textcolor{red}{V}}{e^{\frac{10 - \textcolor{red}{V}}{10}} - 1} $& &
$\textcolor{colorcoeff}{\beta_n(V)}$ &=& $0.125 \  e^{-\frac{\textcolor{red}{V}}{80}}$ \\[5mm]
$\textcolor{colorcoeff}{\alpha_m(V)} $&=& $0.1 \ \frac{25 - \textcolor{red}{V}}{e^{\frac{25 - \textcolor{red}{V}}{10}} - 1}$& &
$\textcolor{colorcoeff}{\beta_m(V)} $&=& $4 \  e^{-\frac{\textcolor{red}{V}}{18}}$ \\[5mm]
$\textcolor{colorcoeff}{\alpha_h(V)} $&=&$ 0.07 \ e^{-\frac{\textcolor{red}{V}}{20}}$ & &
$\textcolor{colorcoeff}{\beta_h(V)} $&=&$ \ \frac{1}{e^{\frac{30 - \textcolor{red}{V}}{10}} + 1}$\\[4mm]
\end{tabular}\\[3mm]
\end{center}
\visible<25>{d'o� notre mod�le complet.}
}
\only<26>{
\raisebox{2cm}{\includegraphics[height=6.5cm]{HHModel/HH.png}}\\
}
\end{frame}


\begin{frame}
\frametitle{Testons le mod�le}
On va tester deux caract�ristiques des vrais neurones que notre mod�le devrait avoir :\\[2mm]
\begin{enumerate}
\item {\normalsize Bifurcation} \visible<2-3>{\\\small{(apparition d'un spike � partir d'une certaine valeur du param�tre $I$),}}\\[2mm]
\item {\normalsize R�sonance} \visible<3>{\\\small{(ce n'est pas parce qu'on augmente la valeur du param�tre $I$ qu'il y a plus de spikes).}}
\end{enumerate}
\end{frame}



\begin{frame}
\frametitle{HH model : test 1}
\vspace{3mm}
 \begin{minipage}{0.3cm}
  
\begin{center}
\visible<2-6>{\noindent\\[0.2cm]
\vspace{0.2cm}$V(t)$\\[0.75cm]
}
\visible<6>{ 
$g_{Na}(t)$\\[0.75cm]
$g_K(t)$\\[0.75cm]
$m(t)$\\[0.75cm]
$h(t)$
}
\end{center} 

\end{minipage}
\begin{minipage}{2.5cm}
  
\begin{center}\small{ I = 2.2}\\[2mm]
\visible<2-6>{\includegraphics[width = 1.8cm]{IBifurc/Iegal2.2/NeuralImpulsesTheActionPotentialInAction-source21} \\
}
\visible<6>{\includegraphics[width = 1.8cm]{IBifurc/Iegal2.2/NeuralImpulsesTheActionPotentialInAction-source22} \\ 
\includegraphics[width = 1.8cm]{IBifurc/Iegal2.2/NeuralImpulsesTheActionPotentialInAction-source23} \\ 
\includegraphics[width = 1.8cm]{IBifurc/Iegal2.2/NeuralImpulsesTheActionPotentialInAction-source25}\\
\includegraphics[width = 1.8cm]{IBifurc/Iegal2.2/NeuralImpulsesTheActionPotentialInAction-source26} 
}
\end{center} 

\end{minipage}
 \begin{minipage}{2.5cm}

\begin{center}  \small{I = 2.2406730\;\;\;\;}\\[2mm]
\visible<3-6>{\includegraphics[width = 1.8cm]{IBifurc/Iegal2.2406730/NeuralImpulsesTheActionPotentialInAction-source21}\\
}
\visible<6>{\includegraphics[width = 1.8cm]{IBifurc/Iegal2.2406730/NeuralImpulsesTheActionPotentialInAction-source22}\\
\includegraphics[width = 1.8cm]{IBifurc/Iegal2.2406730/NeuralImpulsesTheActionPotentialInAction-source23}\\
\includegraphics[width = 1.8cm]{IBifurc/Iegal2.2406730/NeuralImpulsesTheActionPotentialInAction-source25}\\
\includegraphics[width = 1.8cm]{IBifurc/Iegal2.2406730/NeuralImpulsesTheActionPotentialInAction-source26} 
}
\end{center} 

\end{minipage}
\begin{minipage}{2.5cm}
  
 \begin{center} \small{ I = 2.2406731\;\;\;\;}\\[2mm]
\visible<4-6>{\includegraphics[width = 1.8cm]{IBifurc/Iegal2.2406731/NeuralImpulsesTheActionPotentialInAction-source21}\\}
\visible<6>{\includegraphics[width = 1.8cm]{IBifurc/Iegal2.2406731/NeuralImpulsesTheActionPotentialInAction-source22}\\
\includegraphics[width = 1.8cm]{IBifurc/Iegal2.2406731/NeuralImpulsesTheActionPotentialInAction-source23}\\
\includegraphics[width = 1.8cm]{IBifurc/Iegal2.2406731/NeuralImpulsesTheActionPotentialInAction-source25}\\
\includegraphics[width = 1.8cm]{IBifurc/Iegal2.2406731/NeuralImpulsesTheActionPotentialInAction-source26} 
}
\end{center} 

\end{minipage}
\begin{minipage}{2.5cm}

\begin{center}\small{   I = 5\;\;\;\;}\\[2mm]
\visible<5-6>{\includegraphics[width = 1.8cm]{IBifurc/Iegal5/NeuralImpulsesTheActionPotentialInAction-source21}\\}
\visible<6>{%
\includegraphics[width = 1.8cm]{IBifurc/Iegal5/NeuralImpulsesTheActionPotentialInAction-source22}\\
\includegraphics[width = 1.8cm]{IBifurc/Iegal5/NeuralImpulsesTheActionPotentialInAction-source23}\\
\includegraphics[width = 1.8cm]{IBifurc/Iegal5/NeuralImpulsesTheActionPotentialInAction-source25}\\
\includegraphics[width = 1.8cm]{IBifurc/Iegal5/NeuralImpulsesTheActionPotentialInAction-source26} 
}
\end{center} 

\end{minipage}
\end{frame}


\begin{frame}
\frametitle{HH model : test 2}
\begin{center}
\includegraphics[height = 1cm]{Resonator/resoForm.png} \\[5mm]
\visible<1>{\includegraphics[height = 3.3cm]{Resonator/reso.png} }
\end{center}
\end{frame}



\section[Data]{Data}
 
\begin{frame}
\begin{center}
	$\mathcal{DATA}$
\end{center}
\end{frame}

\begin{frame}
	\frametitle{Obtenir des donn�es}
	M�thodes d'obtention de donn�es neuronales en 3 cat�gories :\\[3mm]
	\begin{itemize}
		\item {\visible<2-4>{invasives, }}
		\item {\visible<3-4>{semi-invasives,}}
		\item {\visible<4>{non-invasives.}}
	\end{itemize}
\end{frame}

\subsection{EEG}
\begin{frame}
	\frametitle{EEG/MEG : �lectro/magnetoenc�phalogramme}
	\begin{center}
		\includegraphics[height = 3.6cm]{methods/EEG2bis.png}\\[2mm]
		\includegraphics[height = 3cm]{methods/EEG1.jpg}\hspace{2mm}
		\includegraphics[height = 3cm]{methods/epilSeizure.png}\\[2mm]
		Electrodes qui captent les variations �lectriques/magn�tiques.
	\end{center}
\end{frame}

\subsection{MRI}
\begin{frame}
	\frametitle{MRI : Imagerie par R�sonance Magn�tique}
	\begin{center}	
		\includegraphics[height = 5cm]{methods/MRI.jpg}\\[2mm]
		{\footnotesize Excitation magn�tique des atomes d'hydrog�ne.}
	\end{center}
\end{frame}

\begin{frame}
	\frametitle{fMRI: Imagerie par R�sonance Magn�tique fonctionnelle}
	\begin{center}
		\includegraphics[height = 6cm]{methods/fMRI2.png}\\[2mm]
		{\footnotesize Mesure la quantit� d'oxyg�ne, croissante dans les zones actives.}
	\end{center}
\end{frame}

\begin{frame}
	\frametitle{dMRI: IRM de diffusion}
	\begin{center}
		\includegraphics[height = 5cm]{methods/diffMRI.jpeg}\\[2mm]
		{\footnotesize Utilise le mouvement brownien des mol�cules d'eau du corps humain.}
	\end{center}
\end{frame}


\section[Ref\&Conc]{R�f�rences et Conclusion}

\begin{frame}
\begin{center}
	$\mathcal{LE \;\;MOT\;\; DE\;\; LA\;\; FIN}$
\end{center}
\end{frame}

\subsection{Ref}

\begin{frame}
\frametitle{R�f�rences}
Wikip�dia (toujours v�rifier les sources)\\
Google\\
Twitter\\[2mm]

Denis Le Billan, \textit{Le cerveau de crystal}\\
Izhikevich, \textit{Dynamical Systems in Neuroscience: The Geometry of Excitability and Bursting}\\

\end{frame}


\subsection{Conclusion}

\begin{frame}
\frametitle{Conclusion}
Les sciences se partagent la compr�hension des ph�nom�nes ; �tre scientifique implique une ouverture.\\[2mm]

Et il n'est jamais trop t�t pour �tre chercheur : n'attendez pas que les choses arrivent.
\end{frame}


\begin{frame}
\begin{center}
Merci de votre attention\\[5mm]

Et si vous avez des questions scientifiques : allevity@ihr.mrc.ac.uk
\end{center}
\end{frame}

\end{document}