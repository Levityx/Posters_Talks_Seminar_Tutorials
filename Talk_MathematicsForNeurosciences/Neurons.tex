\begin{frame}\begin{center}
	$\mathcal{NEURONES}$
	\end{center}
\end{frame}



\subsection{}%%%
\begin{frame}
\frametitle{Quelques nombres}
\begin{itemize}
\item $\approx$100.000.000.000 neurones chez l'Homme %\only<2>{\\[2mm]}
\only<1>{\mbox{\phantom{\hspace{3mm}\includegraphics[height=6.9cm]{Neurons/Brains.jpg}}}}
\only<2>{\begin{center}\hspace{3mm}\\[-2mm]\includegraphics[height=6.9cm]{Neurons/Brains.jpg}\end{center}}
\only<3->{\item Jusqu'� 10.000 connections par neurone}
\only<3>{\mbox{\phantom{\hspace{1mm}\includegraphics[height=6.5cm]{Neurons/neuronGrowthDeath.jpg}}}}
%http://www.theemotionmachine.com/mindfulness-and-neuroplasticity
\only<4>{\includegraphics[height=6.5cm]{Neurons/neuronGrowthDeath.jpg}}
\only<5->{\item Plusieurs types de neurones}
\only<5>{\mbox{\phantom{\hspace{1mm}\includegraphics[height=6cm]{Neurons/neurontypes.png}}}}
%http://www.mind.ilstu.edu/curriculum/neurons_intro/imgs/neuron_types.gif
\only<6>{\includegraphics[height=6cm]{Neurons/neurontypes.png}}
\only<7->{\item Diff�rentes �chelles possibles\\}
\only<7>{\mbox{\phantom{\hspace{1mm}\includegraphics[height=5.5cm]{Neurons/lengths.jpg}}}}
%http://thebrain.mcgill.ca/flash/d/d_06/d_06_cl/d_06_cl_mou/d_06_cl_mou_1a.jpg
\only<8>{\begin{center}\includegraphics[height=5.5cm]{Neurons/lengths.jpg}\end{center}}
\end{itemize}
\end{frame}

\subsection{R�seaux}%%%
\begin{frame}
\frametitle{Connectome}
D'o� des connections difficiles � analyser.\\
\begin{center}\includegraphics[height=5.5cm]{Neurons/neural-network-model.jpg}\end{center}
%http://4.bp.blogspot.com/-7roS9m_NcGk/Ui7yhdCTEYI/AAAAAAAAASU/iAK3De3Yahg/s1600/neural-network-model.jpg
\end{frame}

\subsection{Activit� �lectrique}%%%
\begin{frame}
\frametitle{Neurones : deux r�gimes �lectriques}
\begin{center}
Information neuronale �lectrique $\implies$ �lectrodes\\[3mm]
\only<1>{\includegraphics[height=5.5cm]{Neurons/electrode.jpg}}
%http://hirnforschung.kyb.mpg.de/uploads/pics/EN_M1_clip_image002_03.jpg
\only<2->{\includegraphics[height=5.5cm]{Neurons/electrode2.jpg}}
\end{center}
\end{frame}

%Leur demander comment s'occuper de cette somme. 
% Ils donneront des r�ponses physiques (pousser les autres), et leur dire qu'il y a des m�thodes math�matiques (ICA, PCA, ...)

\begin{frame}
\frametitle{Neurones : deux r�gimes �lectriques}
\hspace{-10mm}
\begin{center}
\begin{minipage}{5mm}
\tiny{Voltage (mV)}\\
\begin{tikzpicture}[scale=1]
\draw [->] (-5,-5) -- (-5,0);
\end{tikzpicture}
\end{minipage}
\hspace{-2mm}
\begin{minipage}{10cm}
\only<1>{\includegraphics[width=11cm]{Neurons/Data0_0.png}}
\only<2>{\includegraphics[width=11cm]{Neurons/Data0_1.png}}
\only<3, 4->{\includegraphics[width=11cm]{Neurons/Data0.png}\\}
\only<4>{\includegraphics[width=11cm]{Neurons/Data52.png}}
\only<1, 2, 3>{\mbox{\phantom{\includegraphics[width=11cm]{Neurons/Data52.png}}}}
\end{minipage}

\vspace{-2.mm}
\begin{tikzpicture}[scale=1]
\draw [->] (-2.3,0) -- (7,0);
\end{tikzpicture}
\tiny{Time (ms)}
\end{center}
\end{frame}


\begin{frame}
\frametitle{Dynamique des potentiels d'action}
\begin{center}
Les spikes se ressemblent \\[2mm]
\includegraphics[width=10.5cm]{Neurons/15000pointsofzelle17septthespikes2}
\end{center}
\end{frame}

