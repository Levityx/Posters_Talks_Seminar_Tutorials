
\setbeamercovered{transparent}

\subsection{Mod�liser}

\begin{frame} 
\begin{itemize}
\item Santiago Ram\'on y Cajal, fin du XIX�me si�cle, \\
p�re des neurosciences.\\
\pause
\item Hodgkin-Huxley, 1952\\
analogie entre un neurone et un syst�me �lectrique,\\ 
\pause
\item Aujourd'hui, on essaie de d�chiffrer le code neuronal.\\[5mm]
\pause
 \begin{center}Quels r�les ont les math�matiques l�-dedans ?\hspace{3mm}\\[4mm]
 \visible<5>{\textbf{Mod�liser et analyser.}}
\end{center}
 \end{itemize}
\end{frame}

%Pr�sentation biologique et fonctionnelle du cerveau, de macro � micro
%Pr�sentation de divers outils math�matiques
%Mod�lisation, de micro � macro

\begin{frame}
\frametitle{Mod�liser}
\vspace{3mm}What I cannot create I do not understand.\\[3mm]
\begin{center}\includegraphics[width = 8cm]{feynman.jpg}\end{center}
\hfill Richard Feynman (1918-1988) 
\end{frame}

\subsection[�chelles]{Les �chelles}

\begin{frame}
\frametitle{Les �chelles (scaling)}
Exemple physique : du macroscopique au microscopique.\\

\begin{tikzpicture}
 \visible<2->{ \node (img1) {\includegraphics[width=4cm]{Intro/universe.jpg}};}
  %\pause
 \visible<3->{ \node (img2) at (img1.south east) [xshift=-3mm, yshift=1cm] {\includegraphics[width=5cm]{Intro/earth.jpg}};}
  %\pause
  \visible<4->{ \node (img3) at (img2.south east) [xshift=-3mm,yshift=0.7cm] {\includegraphics[width=4cm]{Intro/human.jpg}};}
  %\pause
  \visible<5->{ \node (img4) at (img3.south east) [xshift=-3mm,yshift=0.3cm] {\includegraphics[width=3.5cm]{Intro/Cell.jpg}};}
  %\pause
  \visible<6->{ \node (img5) at (img4.south east) [xshift=-3mm,yshift=0.9cm] {\includegraphics[width=3.5cm]{Intro/water.jpg}};}
 
  \end{tikzpicture}

La physique th�orique cherche encore une th�orie unificatrice.\\
%http://htwins.net/scale2/\\
%http://micro.magnet.fsu.edu/primer/java/scienceopticsu/powersof10/index.html
\end{frame}

\begin{frame}
\frametitle{Les �chelles (scaling)}
Exemple Math�matique : un monde sans �chelles.\\
Les fractales (auto-similaires) \\[3mm]
\only<1>{\mbox{\phantom{\includegraphics[width=3.5cm]{Droste/Sierpinski.png}}}}%
\includegraphics<2->[width=3.5cm]{Droste/Sierpinski.png}\hspace{1mm}%
\only<-2>{\mbox{\phantom{\includegraphics[width=3.8cm]{Droste/escher.jpg}}}}%
\includegraphics<3->[width=3.8cm]{Droste/escher.jpg}\hspace{1mm}%
\only<-3>{\mbox{\phantom{\includegraphics[width=3cm]{Droste/00.jpg}}}}%
\includegraphics<4->[width=3cm]{Droste/00.jpg}\hspace{1mm}%
\only<-4>{\mbox{\phantom{(Escher, Droste effect, Sierpinski)}}}%
\only<5->{(Sierpinski, \hspace{1.7cm}Escher, \hspace{2.6cm}Droste effect)}
\end{frame}

\begin{frame}
\frametitle{Les �chelles (scaling)}
Exemple Math�matique : un monde d'�chelles.\\
Les fractales (non auto-similaires)\\[3mm]
\begin{center}
\only<1>{\mbox{\phantom{\includegraphics[width=9.5cm]{Fractals/Mandel.jpg}}}}%
\includegraphics<2->[width=9.cm]{Fractals/Mandel.jpg}
\end{center}
\only<-2>{\mbox{\phantom{http://www.htwins.net/mandyzoom/http://www.youtube.com/watch?v=tzNjmSGVs6o}}}
\only<3->{http://www.htwins.net/mandyzoom/\\http://www.youtube.com/watch?v=tzNjmSGVs6o}
\end{frame}

\begin{frame}
\frametitle{Les �chelles (scaling)}
Exemple Neuro : du macroscopique au microscopique.\\
\begin{tikzpicture}
 \visible<1->{ \node (img1) {\includegraphics[width=4.5cm]{BrainScales/1_brain.png}};}
  \pause
 \visible<2->{ \node (img2) at (img1.south east) [yshift=0.5cm] {\includegraphics[width=4.8cm]{BrainScales/2_brain.jpg}};}
  \pause
  \visible<3->{ \node (img3) at (img2.south east) [yshift=0.5cm] {\includegraphics[width=4cm]{BrainScales/3_vibrissal-cortex-rat.jpg}};}
  \pause
  \visible<1->{ \node (img4) at (img3.south east) [yshift=0.5cm] {\includegraphics[width=5cm]{BrainScales/4_neurons.png}};}
  \pause
  \visible<5->{ \node (img5) at (img4.west) [xshift=-2cm, yshift=-0.8cm] {\includegraphics[width=3cm]{BrainScales/6_Ionic.jpg}};}
 % \pause
   \visible<4->{\node (img6) at (img4.north) [xshift=0.4cm, yshift=1.5cm] {\includegraphics[width=3cm]{BrainScales/5_Synapse.png}};}
  \end{tikzpicture}
  \end{frame}

\subsection{Technologie}


\begin{frame}
\frametitle{D�couvertes et technologie}
\begin{tabular}{ c c c }
  Santiago Ram\'on y Cajal, 1899 && Lynne Quarmby, 2011 \\[3mm]
  \includegraphics[height=4cm]{DecouvertesTechn/1899_PurkinjeCell.jpg} && \includegraphics[height=4cm]{DecouvertesTechn/2011_cell.jpg} \\
\end{tabular}
\end{frame}