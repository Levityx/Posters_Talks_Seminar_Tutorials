
\begin{frame}
\begin{center}
	$\mathcal{DATA}$
\end{center}
\end{frame}

\begin{frame}
	\frametitle{Obtenir des donn�es}
	M�thodes d'obtention de donn�es neuronales en 3 cat�gories :\\[3mm]
	\begin{itemize}
		\item {\visible<2-4>{invasives, }}
		\item {\visible<3-4>{semi-invasives,}}
		\item {\visible<4>{non-invasives.}}
	\end{itemize}
\end{frame}

\subsection{EEG}
\begin{frame}
	\frametitle{EEG/MEG : �lectro/magnetoenc�phalogramme}
	\begin{center}
		\includegraphics[height = 3.6cm]{methods/EEG2bis.png}\\[2mm]
		\includegraphics[height = 3cm]{methods/EEG1.jpg}\hspace{2mm}
		\includegraphics[height = 3cm]{methods/epilSeizure.png}\\[2mm]
		Electrodes qui captent les variations �lectriques/magn�tiques.
	\end{center}
\end{frame}

\subsection{MRI}
\begin{frame}
	\frametitle{MRI : Imagerie par R�sonance Magn�tique}
	\begin{center}	
		\includegraphics[height = 5cm]{methods/MRI.jpg}\\[2mm]
		{\footnotesize Excitation magn�tique des atomes d'hydrog�ne.}
	\end{center}
\end{frame}

\begin{frame}
	\frametitle{fMRI: Imagerie par R�sonance Magn�tique fonctionnelle}
	\begin{center}
		\includegraphics[height = 6cm]{methods/fMRI2.png}\\[2mm]
		{\footnotesize Mesure la quantit� d'oxyg�ne, croissante dans les zones actives.}
	\end{center}
\end{frame}

\begin{frame}
	\frametitle{dMRI: IRM de diffusion}
	\begin{center}
		\includegraphics[height = 5cm]{methods/diffMRI.jpeg}\\[2mm]
		{\footnotesize Utilise le mouvement brownien des mol�cules d'eau du corps humain.}
	\end{center}
\end{frame}
