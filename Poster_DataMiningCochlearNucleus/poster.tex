\documentclass[20pt, a0paper, portrait,innermargin=10mm,
blockverticalspace=15mm, colspace=15mm, subcolspace=8mm]{tikzposter} 
% Author: Alban.Levy@nottingham.ac.uk
% Poster made in September 2015, presenting work in progress

 %%%%%%%% %%%%%%%% %%%%%%
 % 				Packages 			   %
 %%%%%%%% %%%%%%%% %%%%%%
 

\usepackage{enumitem} % to change labels in enumerate 
\usepackage{graphicx,float} % to play with logos
\usepackage{tikz}
\usetikzlibrary{shadings}
\usepackage{amsmath}


 %%%%%%%% %%%%%%%% %%%%%%
 % 				Paths 			   %
 %%%%%%%% %%%%%%%% %%%%%%
 
 
 % Since I work from various computers sharing Dropbox, 
 % I've defined some paths in  file Dropbox/pathsForLaTeXFiles.tex
 % Obviously I check which computer by verifying RxPz logo is there
 \IfFileExists{/Users/pmxal9/Dropbox/Logos/RxPz.jpg}{
	\def\computer{/Users/pmxal9/} 
}{	
	\def\computer{/Users/pmaal/} 
}
\edef\drop{\computer Dropbox/}
\input{\drop pathsForLaTeXFiles.tex}
\edef\Pics{\FYR Pics/}
\def\SET{\drop Nottingham/SET/}

% Logos
\edef\logoNotts{\drop Logos/notts.png}
\edef\logoIHR{\drop Logos/ihr.png}
\edef\logoNETT{\drop Logos/NETT.png}
\edef\logoPerso{\drop Logos/Logo.jpg}
\edef\logoRxPz{\drop Logos/RxPz.jpg}
\edef\logoEC{\drop Logos/EC.jpg}
\edef\logoMCA{\drop Logos/MCA.jpg}


 %%%%%%%% %%%%%%%% %%%%%%
 % 				Other 			   %
 %%%%%%%% %%%%%%%% %%%%%%

 
% Changed answerToChris folder momentarily
%\edef\ansChris{\ansChris ../answersToChris_until_2015Sept15th/}

% Use a counter for the blocks
\newcounter{counter}
\setcounter{counter}{1} 
% Function to add counter to block title and increment its value
\newcommand*\counterInc{\circled{\arabic{counter}}\addtocounter{counter}{1}}

% Use a counter for references
\newcounter{ref}
\setcounter{ref}{1} 
% Function to print ref counter and increment its value
\newcommand*\refInc{{\arabic{ref}}\addtocounter{ref}{1}}

% Command to enumerate blocks
\newcommand*\circled[1]{\tikz[baseline=(char.base)]{%
            \node[shape=circle,draw,inner sep=2pt] (char) {#1};}\,\,}
        
 % turn comment off
\tikzposterlatexaffectionproofoff        

% Add stuff to do
\newcommand*\addStuffToDo[1]{\fbox{%
	\begin{minipage}{0.94\colwidth}{\textcolor{red}{#1}}\end{minipage}}}


 %%%%%%%% %%%%%%%% %%%%%%
 % 			      Title 	Logos 		   %
 %%%%%%%% %%%%%%%% %%%%%% 


 % TITLE
 \title{\Huge Envelope coding in the Cochlear Nucleus: a Data Mining Approach}

% SYMBOLS to use for the adresses and authors
\def\characA{\oplus} 		
\def\characB{\ominus}	
\def\characC{\otimes}	

% AUTHORS
 \author{\Large 
 	Alban Levy${}^{\characA\characB}$  
	Stephen Coombes${}^\characA$ 
	Christian Sumner${}^\characB$
	Aristodemos Pnevmatikakis${}^\characC$%
}   

% INSTITUTES
\institute{\large University of Nottingham:\hspace{4mm}%
	School of Mathematical Sciences${}^\characA$\hspace{4mm}%
	Institute of Hearing Research${}^\characB$ \hspace{4mm}%
	Athens Information Technology${}^\characC$\\[0.5cm]
	\texttt{Alban.Levy@nottingham.ac.uk}\\[-15mm]
}  

 % Function for logos
\newcommand\includelogo[1]{\includegraphics[height=2.8cm]{#1}}
\newcommand\includelogospace[1]{\includelogo{#1}\hspace{40mm}}
\titlegraphic{%
	\includelogospace{\logoNotts} 
	\includelogospace{\logoIHR} 
 	\includelogospace{\logoPerso} 
 	\includelogospace{\logoRxPz} 
	\includelogospace{\logoNETT}
	\includelogospace{\logoMCA} 
	\includelogo{\logoEC}
}

% Theme
\usetheme{Basic}  
\usecolorstyle{Australia}
\colorlet{backgroundcolor}{white}
 
 
 %%%%%%%% %%%%%%%% %%%%%%%%
 %				Document				 %
 %%%%%%%% %%%%%%%% %%%%%%%%
 
 
 \begin{document}
     \maketitle[titletotopverticalspace=0.5cm,titlegraphictotitledistance=1cm] 
     
     \block[titlewidthscale=0.5]{Introduction}{
	\begin{minipage}{39.2cm}
		The Cochlear Nucleus (CN) processes the auditory information received from the inner hair cells in the cochlea, 
		directly via the auditory nerve.  It is well established  that the timing of spikes in the CN precisely reflect the 
		fine structure and envelope of sounds. Conventionally it has been assumed that the relationship between 
		the stimulus and the timing of spikes can be approximated as linear. Classically the Vector Strength (VS) 
		has been used to evaluate the information contained in the spike trains. Such analysis leads to a 
		particular interpretation of temporal coding in the CN. 	
 	\end{minipage}%
	\begin{minipage}{2cm}
	\hspace{2cm}
	\end{minipage}%
	\begin{minipage}{39.1cm}
		Recent findings suggest that this measure is not universally appropriate as it does not 
		accurately reflect the information contained in the spike trains. 
		To extract as much information as we can from the spike trains recorded and with less assumptions, 
		we developed a computational architecture to use modern tools of Data Mining as a means to extract the modulation frequency 
		of an Amplitude-Modulated tone from the patterns of spikes produced by CN neurons, using a supervised learning framework. 
		We present the results of these computations and conclusions we can draw about processing in the CN.
	\end{minipage}%
   }  
	
     \begin{columns}  
         \column{0.45}  
         \block[titlewidthscale=0.45]{\counterInc CN Spike Trains}{%
		\begin{minipage}{12.2cm}
			\vspace{-6mm}
			\includegraphics[width=12.2cm, height=12.2cm,trim=0cm 0.4cm 0cm 2.5cm, clip=true]{\ansChris  Classi2_copy1.png}
			\vspace{-17mm}
		\end{minipage}%
		\begin{minipage}{22.4cm}
			\def\modLevel{M}
			\def\fmod{f_m\,}
			\def\fcarr{f_c\,}
			$\bullet$
			\textbf{Raster plot} of  300 spike trains: responses of one cochlear nucleus neuron in stationary state over 80ms. \\[-5mm]
			
			$\bullet$
			For each modulation frequency $\fmod \in \{50, 150,\dots,1150\}$Hz, 
			an \textbf{amplitude modulated tone} is played 25 times to the anaesthetised animal:
			$$s(t) = \modLevel\left(1+\sin(2\pi \fmod t)\right)\sin(2\pi\fcarr t)$$
			\noindent for fixed modulation level $\modLevel$ and carrier frequency $\fcarr$.\\[-5mm]
			
			$\bullet$
			\textbf{463} such datasets with 300 spike trains.\\[-5mm]
			
			$\bullet$
			Cats' cochlear nucleus data recorded by Rhode \& Greenberg~[\refInc].
		\end{minipage}
	}
	
	\newlength\cwidth
	\newlength\cheight
	\setlength\cwidth{30cm}
	\setlength\cheight{14cm}

	\block[titlewidthscale=0.45]{\counterInc Vector Strength}{
		\hspace{-0.6cm}
		\begin{minipage}{16.5cm}
			\vspace{-4mm}
			\edef\VectorStrength{\ansChris  plotVectorStrength46_AM_headLength1_HeadWidth1_7.pdf}
			\includegraphics[width=16cm,trim=2.1cm 1.2cm 0.5cm 13.2cm, clip=true]{\VectorStrength} 
		\end{minipage}%
		\begin{minipage}{18.7cm}
				Vector Strength [\refInc] computation on 3 spike trains from an Onset unit, knowing the input (grey AM tone): 
				each spike is associated with a phase $\phi_k$ (phase-coloured). \\
			
				The VS  $\rho\in[0,1]$ is the length of the black arrow:
				$$\rho = \left|\frac{1}{n}\sum_{k=1}^n \text{e}^{i\phi_k}\right|.$$
		\end{minipage}\\[3mm]%
		The transfer modulation transfer function (tMTF) is a trace of modulation gain as a function of modulation frequency. 
		It characterises the range of modulation frequencies a neuron encodes.~Cf~[1]: \\
		$\bullet$ PL units are low pass envelope encoders like the AN. \\
		$\bullet$ Choppers show modulation tuning:  bandpass tMTF (higher VS than PLs and a peak VS).\\
		$\bullet$ Onset units are the best at encoding periodicity. \\[3mm]
		\begin{minipage}{18cm}
			\fbox{\includegraphics[width=18cm,trim=0cm 0cm 0cm 0cm, clip=true]{\Thesis DataMiningCochlearNucleus/Rhode1_2.png}}\\
			\fbox{\includegraphics[width=18cm,trim=0cm 0cm 0cm 0cm, clip=true]{\Thesis DataMiningCochlearNucleus/Rhode2_2.png}}
		\end{minipage}
		\begin{minipage}{16cm}
			\fbox{\includegraphics[width=15.465cm,trim=0cm 0cm 0cm 0cm, clip=true]{\Thesis DataMiningCochlearNucleus/Rhode3_4.png}}
		\end{minipage}
	}
	
	\block[titlewidthscale=0.35]{\counterInc  Preprocessing}{
		\vspace{-2cm}\hspace{-7mm}
		\begin{minipage}{17.5cm}
			{\centering \includegraphics[width=17cm,trim=4.3cm 0cm 4cm 3cm, clip=true]{\DecisionTree NewTree.pdf}}
			\vspace{3mm}
		\end{minipage}%
		\begin{minipage}{17.1cm}
		\vspace{10mm}
			Each dataset is preprocessed in 3 different ways: features from the interspike intervals, time-binning, or using a spike metric [\refInc]. 
			We then apply 
			all available classification algorithms from the data mining toolbox Weka [4]. \\
		\end{minipage}%
		
		\hspace{-9mm}
		\begin{tabular}{c|c|c}\hline\vspace{-1mm}
		\textbf{ISI Features} & \textbf{Time-Binning} & \textbf{Spike Metric}\\
			\begin{minipage}{0.3202\colwidth}
				Mean, variance,  coefficient of variation, skewness, number of spikes, histogram values, \dots
			\end{minipage}&
			\begin{minipage}{0.3202\colwidth} 
				\vspace{-10mm}
				\includegraphics[width=0.32\colwidth,trim=0.2cm 0cm 0cm 0cm,clip=true]{\Thesis DataMiningCochlearNucleus/TimeBinning.pdf}
			\end{minipage}&
			\begin{minipage}{0.3202\colwidth}
				\vspace{-8mm}
				\includegraphics[width=0.32\colwidth,trim=5.3cm 18.5cm 3cm 3cm,clip=true]{\Thesis DataMiningCochlearNucleus/VictorPurpura.pdf}
			\end{minipage}
		\end{tabular}	
		\vspace{-8mm}
	}
	
	% Get the blocks closer
	\newlength\closeblock
	% Difference between the two ratios of block width
	\setlength\closeblock{0.05\colwidth} 
	\newlength\lilspace
	\setlength\lilspace{-8mm}
	
         \block[titlewidthscale=0.45]{\counterInc Supervised Learning}{
		\begin{minipage}{0.24\colwidth}
			\vspace{-14mm}
			\includegraphics[width=0.25\colwidth,trim=0cm 0cm 0cm 0cm, clip=true]{\ansChris  Classi2_copy2.png}
		\end{minipage}%
		\begin{minipage}{0.72\colwidth}{
			$\bullet$ Typical \textbf{confusion matrix} as output by \textbf{Weka} [\refInc].\\[\lilspace]

			$\bullet$ 
			$\#(i,j)$ = \#instances of \textbf{class $i$ {classified} as belonging to class $j$.}\\[\lilspace]
			
			$\bullet$
			Algorithms evaluated by stratified 10-fold \textbf{cross-validation}.\\[\lilspace]
	
			$\bullet$
			Spike trains corresponding to low $f_m$ are easily classified. \\[\lilspace]
			
			$\bullet$
			At high $f_m$, the algorithm is mostly guessing.\\[\lilspace]
			
			$\bullet$
			\textbf{Low-pass tMTF} in this chopper unit for AM frequency transmission. 
		}
		\end{minipage}\\[-7mm]
         }
        
	
	%%%%%%%%%%%% COLUMN 2
         \column{0.55}

         \block[titlewidthscale=0.75]{\counterInc Average accuracy per classifier and unit type}{%
		\hspace{-5mm}
			\begin{minipage}{\colwidth}
				\edef\allMean{\ansChris forAllClassifierAllTypes_mean_pctCorrect_surf_sortedTwoSMO_8.pdf}
				\includegraphics[width=0.97\colwidth, trim=5cm 3.5cm 3cm 1cm, clip=true]{\allMean}
			\end{minipage}\\%
			\begin{minipage}{0.97\colwidth}
				$\bullet$ 
				Mean accuracy (percentage of correct guesses, $z$-axis) of all classification algorithms ($x$-axis) 
				and unit types ($y$-axis). \\
				%
				$\bullet$ 
				Both  \textbf{algorithms} and \textbf{unit types are overall ordered}: 
				no preference from some algorithms to specific unit types. \\
				%
				$\bullet$ 
				Unit types differ in their ability to transmit AM frequencies. 
				Overall we confirm previous findings that \textbf{regular} firing cells are good at encoding envelopes. We obtain the order 
				$On > Ch > LowF > PLN > PBU>PL$
				while [1] obtain the following rank order for phase-locking capability:
				$PLN>(On=PL)>Ch>PBU$.\\
				%
				$\bullet$
				Different measures of performance generally yield the same relative performance, but different absolute performance.
			\end{minipage}
	}

	\block[titlewidthscale=0.65]{\counterInc Comparing preprocessing methods.}{
		\begin{minipage}{0.48\colwidth}
		\edef\histSMOParulaPdf{\ansChris SMOperParam_pctCorrect_mean_parula_fifthVerReord.pdf} 
		\vspace{-1.58mm}
		\includegraphics[width=0.455\colwidth,trim=1.4cm 0.5cm 1.5cm 0.5cm, clip=true]{\histSMOParulaPdf} \\[-10mm]
		\end{minipage}%
		\begin{minipage}{0.48\colwidth}
			$\bullet$ 
			Figure shows mean performance across all recordings from a given unit type for the SMO classifier, 
			with each preprocessing method. \\
			Left: ISI features. 
			Centre: V\&P spike metric. Right: Time-binning.\\[-9mm]
					
		 	$\bullet$ 
			Renormalised histograms of the best parameter across datasets are shown in grey in each box.\\[-9mm]
			
			$\bullet$  
			SpkM and time-binning give very similar peak values, which correspond to similar time resolutions. \\[-9mm]
			
			$\bullet$  
			In general both SpkM and time binning outperform summary ISI statistics. 
		\end{minipage}
	}
         
        \begin{subcolumns}
		\subcolumn{0.5}
		\block[titlewidthscale=0.75]{\counterInc Transfer Function}{
			\begin{minipage}{14.5cm}
				\edef\cPLot{\ansChris tMTF_max_ClassifForEachType.pdf}
				\includegraphics[width=0.9\subcolwidth,trim=0cm 0cm 0cm 0cm, clip=true]{\cPLot} 
			\end{minipage}\\[3mm]%
			\begin{minipage}{0.92\subcolwidth}
				Mean percentage correct classification vs modulation frequency for each unit type, 
				using  SMO classifier and spike-metric preprocessing.\\[-7mm]
				
				Stars show the cut-off frequencies $$H(\omega_{co})=||H||_\infty/\sqrt{2}.$$
				\noindent\\[-12mm]
				Curves are low-pass for all unit types. \\[-7mm]
				
				Primary-Like units' slope is low.\\[-10.9mm]
			\end{minipage}%
		}
		
		\setlength\lilspace{-6.35mm}	
		\subcolumn{0.5}
	
		\block[titlewidthscale=0.85]{\counterInc Optimal time-scales}{%
			\begin{minipage}{0.9\subcolwidth}
				\edef\optimalTS{\ansChris OptimalTimeStepVSAccuracy_hist.pdf}
				\includegraphics[width=0.9\subcolwidth, trim=0.5cm 0.3cm 0.4cm 0.5cm, clip=true]{\optimalTS}\\[-10mm]%
			\end{minipage}\\%
			\hspace{-6mm}
			\begin{minipage}{0.945\subcolwidth}
				Figure shows, for each recording, the \textbf{optimal time bin resolution vs accuracy} for classification using~SMO.\\[\lilspace]
				
				\textbf{Optimal time-step is sub-millisecond.}\\[\lilspace]
				
				Units where spikes were putatively less well timed gave low performance.\\[\lilspace]
				
				Subtle differences between units: PLN preferring smallest time windows. i.e. being most precise. 
			\end{minipage}%
		}
	\end{subcolumns}
	
	\block[titlewidthscale=0.25]{}{%
		\newcommand{\notation}[1]{[#1]:}
		\vspace{-15mm}
		\begin{minipage}{21cm}
			\begin{enumerate}[label=\notation{\arabic*}]
			\item W. S. Rhode and S. Greenberg. Encoding of amplitude modulation in the cochlear nucleus of the cat.
				 Journal of Neurophysiology, 
				May 1994.\\[-1.5mm]
			\item J. Leo van Hemmen. Vector strength after Goldberg, Brown, and von Mises: biological and mathematical perspectives. 
				Biological Cybernetics, August 2013.
			\end{enumerate}
		\end{minipage}\hspace{3mm}%
		\begin{minipage}{21cm}
			\begin{enumerate}[label=\notation{\arabic*}]
			\setcounter{enumi}{2} 
			\item Jonathan D. Victor and Keith P. Purpura. Metric-space analysis of spike trains: theory, algorithms and application. 
				Network: Computation in Neural Systems, January 1997.\\[-1.5mm]
			\item Mark Hall, Eibe Frank, Geoffrey Holmes, Bernhard Pfahringer, Peter Reutemann, and Ian H. Witten. The WEKA Data 
				Mining Software: An Update. SIGKDD Explor. Newsl., November 2009.
			\end{enumerate}
		\end{minipage}
		\vspace{-3mm}
	}
	\end{columns}
	
\end{document}