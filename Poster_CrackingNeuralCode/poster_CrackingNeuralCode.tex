\documentclass[paperwidth=75cm,paperheight=116.18cm,fontscale=0.40]{baposter}
% Written by Alban.Levy@nottingham.ac.uk

 %%%%%%%% %%%%%%%% %%%%%%
 % 				Packages 			   %
 %%%%%%%% %%%%%%%% %%%%%%
% \usepackage[cmyk]{xcolor}
\usepackage{pgfplots}


\usepackage{enumitem} 		% to change labels in enumerate 
\usepackage{graphicx,float} 	% to play with logos
\usepackage{tikz}
\usetikzlibrary{shadings}
\usepackage{amsmath}
%http://tex.stackexchange.com/questions/88429/color-box-with-rounded-corners-around-a-fragment-of-a-formula
\usepackage[framemethod=TikZ]{mdframed} 
\mdfdefinestyle{MyFrame}{%
    linecolor=blue,
    outerlinewidth=0.5pt,
    roundcorner=5pt,
    innertopmargin=7pt,%\baselineskip,
    innerbottommargin=7pt,%\baselineskip,
    innerrightmargin=15pt,
    innerleftmargin=15pt,
    backgroundcolor=white!50!white
}


% Packages from Baposter template
\usepackage{calc}
\usepackage{graphicx}
\usepackage{amsmath}
\usepackage{amssymb}
\usepackage{relsize}
\usepackage{multirow}
\usepackage{rotating}
\usepackage{bm}
\usepackage{url}
\usepackage{graphicx}
\usepackage{multicol}
\usepackage{tcolorbox}
\usepackage{fancybox}
%\usepackage{pstricks}

% Font package
%\usepackage{fontspec}
%\setmainfont{Calibri}
%\setsansfont{Calibri}

 %%%%%%%% %%%%%%%% %%%%%%
 % 				Paths 			   %
 %%%%%%%% %%%%%%%% %%%%%%
 % Since I work on various computers sharing Dropbox, 
 % I've defined some paths in the file Dropbox/pathsForLaTeXFiles.tex
 % Obv, I check which computer by verifying RxPz logo is there
 \IfFileExists{/Users/pmxal9/Dropbox/Logos/RxPz.jpg}{
	\def\computer{/Users/pmxal9/} 
}{	
	\def\computer{/Users/pmaal/} 
}
\edef\drop{\computer Dropbox/}
\input{\drop pathsForLaTeXFiles.tex}
\edef\Pics{\FYR Pics/}
\def\SET{\drop Nottingham/SET/}

% Logos
\edef\logos{\drop LogoNoBackground/}
\edef\logoNotts{\logos  notts.png} %UoN.png}% looks better without
\edef\logoIHR{\logos ihr_3.png}
\edef\logoNETT{\logos NETT.png}
\edef\logoPerso{\logos Logo.png}
\edef\logoRxPz{\logos RxPz.png}
\edef\logoEC{\logos EC.png}
\edef\logoMCA{\logos MCA.png}

%%%% Paths to pictures
\edef\cochImp{\SET Implants/Blausen_CochlearImplant_02.png} %_01 with 'Cochlear Implant' written
\edef\brainstem{\SET /Implants/ABIHowitWorksImage_1.jpeg}
\edef\dataRep{\SET dataRep3.png}
\edef\cakeSpec{\SET cakeIsLieSpec.png}
\edef\cakeReg{\SET cakeIsLieSpecReg.png}
\edef\PicBrainstem{\SET Pic_brainstemBis.png}
\edef\xy{\SET xy.png}
\edef\xyM{\SET xyMean.png}
\edef\xyMS{\SET xyMeanSplit.png}
\edef\xyMSC{\SET xyMeanSplitCol.png}
\edef\cochg{\SET soundPlots/ySum_cochleo.pdf}
\edef\specg{\SET soundPlots/ySum_spectro.pdf}
\edef\cake{\SET soundPlots/ySum_plot.pdf}


 %%%%%%%% %%%%%%%% %%%%%%
 % 				Other 			   %
 %%%%%%%% %%%%%%%% %%%%%%

% Use a counter for the blocks
\newcounter{counter}
\setcounter{counter}{1} 
% Function to add counter to block title and increment its value
\newcommand\counterInc{\circled{\arabic{counter}}\addtocounter{counter}{1}}

% Use a counter for references
\newcounter{ref}
\setcounter{ref}{1} 
% Function to print ref counter and increment its value
\newcommand\refInc{{\arabic{ref}}\addtocounter{ref}{1}}

% Command to enumerate blocks
\newcommand*\circled[1]{\tikz[baseline=(char.base)]{%
            \node[shape=circle,draw,inner sep=2pt] (char) {#1};}\,\,}

% Add stuff to do
\newcommand\addStuffToDo[1]{\fbox{\begin{minipage}{0.94\colwidth}{\textcolor{red}{#1}}\end{minipage}}}

 %%%%%%%% %%%%%%%% %%%%%%
 % 			      Title 	Logos 		   %
 %%%%%%%% %%%%%%%% %%%%%%   

 % Function for logos
 \newcommand\includelogo[1]{\includegraphics[height=1.1cm]{#1}}
 \newcommand\includelogospace[1]{\includelogo{#1}\hspace{18mm}}
 \newcommand\titlelogos{%
%\begin{mdframed}[style=MyFrame]
	\includelogospace{\logoNotts} 
	\includelogospace{\logoIHR} 
 	\includelogospace{\logoPerso} 
 	\includelogospace{\logoRxPz} 
	\includelogospace{\logoNETT}
	\includelogospace{\logoMCA} 
	\includelogo{\logoEC}
%\end{mdframed}
}

 %%%%%%%%%%%%%%%%%%%%%%%%
 %				Document			        %
 %%%%%%%%%%%%%%%%%%%%%%%%

 \begin{document}
 
% SYMBOLS to use for the adresses and authors
\def\characA{\oplus} 		%\ast %\mizu
\def\characB{\ominus}	%\dagger %\omega
\def\characC{\otimes}	%\dagger %\omega

% Color for title and bottom text. Put white (1,1,1) if very high background opacity 
\definecolor{colorTitle}{rgb}{0,0,0} 

% Background picture, with low opacity to make it whiter
\background{%
	\begin{tikzpicture}
		[remember picture,overlay]\node[opacity=0.1] at (current page.center) {\includegraphics[height=\paperheight]{\SET background3.jpg}}; 
	\end{tikzpicture}%
}

% Frame's headerColorTwo
\definecolor{lightblue}{rgb}{0.145,0.6666,1}
\definecolor{lightred}{rgb}{1,0.6666,0.145}

%% Poster options
 \begin{poster}%
  {
  % Show grid to help with alignment
  grid=false,
  % Column spacing
  colspacing=1em,
  % Color style
  bgColorOne=white,
  bgColorTwo=white,
  borderColor=lightblue!50!white,
  headerColorOne=blue!80!black, %blue!80!white% black
  headerColorTwo=lightblue,
  headerFontColor=white,
  boxColorOne=white,
  boxColorTwo=lightblue,
  % Format of textbox
  textborder=rounded, %roundedleft,
  % Format of text header
  eyecatcher=true,
  headerborder=closed,
  headerheight=0.1\textheight,
  %textfont=\sc, An example of changing the text font
  headershape=rounded, %roundedright
  headershade=shadelr,
  headerfont=\Large\bf\textsc, %Sans Serif
  textfont={\setlength{\parindent}{1em}},
  boxshade=plain,
  %background=shade-tb,
  background=user,
  linewidth=2pt
  }%
{}%
{%
\bf\textsc{\Huge \textcolor{colorTitle}{Cracking\; the\; Neural\; Code}}
}%
 % AUTHORS
{%
\textcolor{colorTitle}{
	\Large \textsc{Alban Levy}${}^{\characA\characB}$  \;\;
	Christian Sumner${}^\characB$ \;\;
	Stephen Coombes${}^\characA$ \;\;
	Aristodemos Pnevmatikakis${}^\characC$
}
%University of Nottingham: %\hspace{4mm}% \large
%	Institute of Hearing Research${}^\characB$ %\hspace{4mm}%
%	School of Mathematical Sciences${}^\characA$%\hspace{4mm}%
%	Athens Information Technology${}^\characC$%[0.5cm]
	\\[6mm]\titlelogos
}{}  


% Bullet style
\newcommand\bul{\noindent$\bullet$\hspace{1mm}}

% Command to enumerate blocks
\newcommand*\circld[1]{\tikz[baseline=(char.base)]{%
            \node[shape=circle,draw,inner sep=2pt] (char) {#1};}\,\,}
        
% Use a counter for the blocks
\newcounter{counterBlock}
\setcounter{counterBlock}{1} 
% Function to add counter to block title and increment its value
\newcommand\counterIncrem{\raisebox{0.15em}{\small\circld{\arabic{counterBlock}}\addtocounter{counterBlock}{1}}}

% Circling block value
\newcommand\ccircled[1]{\small{\circled{#1}}}

% Block lengths
\newlength\blocLeft
\setlength\blocLeft{27.0cm}
\newlength\spaceItem
\setlength\spaceItem{-4.6mm}

\newlength\cwidth
\newlength\cheight
\setlength\cwidth{9.0cm} 
\setlength\cheight{5cm} 



%%%%%%%%%%%%%%%%%%%%%%%%%%%%%%%%%%%%%%%%%%%%%%%%%%%%%%%%%%%%%%%%%%%%%%%%%%%%%%	 
\begin{posterbox}[name=auditoryNeuro,column=1, span=2,row=0,headerColorOne=red!40!black,headerColorTwo=red!75!white,borderColor=red!50!white, textborder=roundedsmall]{ Auditory Neurosciences: \small{Hearing, Implants \&  Neurosciences in a Nutshell}} 
	\noindent\begin{minipage}{10.6cm}
		\vspace*{-1mm}
		\begin{tcolorbox}[colback=red!0!white,colframe=red!90!white,title=\LARGE{\textsc{Our ultimate goal is to}}] 
			\hspace{-4.0mm}
			\begin{minipage}{10.2cm}
			\large
			 {Understand how our brains are able to \textbf{segregate and group elements of sound}  
			to form perceptual objects.}
			\end{minipage}
		\end{tcolorbox} 
		%\begin{tcolorbox}[coltitle=black,colback=blue!0!white,colframe=red!15!white,title=Poster Roadmap: Auditory Neuroscience in a Nutshell]
		%Poster Roadmap \\
		%\noindent\hspace{-2mm}\textcolor{red!70!white}{Block\#}
		%\vspace*{-1.3mm}\noindent\hspace{-3.5mm}{\ovalbox{\textcolor{red!70!white}{Block}\raisebox{0.5mm}{\#}}}\\[-4.8mm]
		\vspace*{-1.42mm}\noindent\begin{itemize}%[leftmargin=7pt, rightmargin=0pt] %\itemsep1pt \parskip0pt \parsep2pt
			\item[\ccircled{1}] We hear mixtures of \textbf{sounds} from individual objects. 
						Still, we can understand speech even when several people are speaking at the same time. %Somehow
				\\[\spaceItem]
			\item[\ccircled{2}] The cochlea \textbf{transduces} vibrations in the air into electrical impulses, \textbf{spikes}, along the hearing nerve fibres. 
			    	\\[\spaceItem]
			\item[\ccircled{3}]  A \textbf{cochlear implant} can be used to excite an impaired cochlea.
				\\[\spaceItem].
			\item[\ccircled{4}]  The spikes from the auditory nerve pass into the brain where they are \textbf{processed} in a number of discrete brain nuclei.
				 \\[\spaceItem] 
			\item[\ccircled{5}] Neurons use spikes in some secret Morse code Biology decided long ago: the\,
						\textbf{neural code},\,inherently\,containing\,\textbf{randomness}. 
						Since neurons transmitting sounds also process it, by building \textbf{computational models}  %or computer model?
						that simulate this processing we can test and develop our understanding of how the code is made.
				 \\[\spaceItem] 
			\item[\ccircled{6}]  A main goal of Neuroscience is to  \textbf{decipher} this code. This requires heavy Maths and \textbf{statistics} machinery.\\
		\end{itemize}
		%\end{tcolorbox} 
	\end{minipage}%
	\begin{minipage}{0.2cm}
		\hspace{0.2cm}
	\end{minipage}%
	\begin{minipage}{7.90cm}
		\begin{tcolorbox}[coltitle=black,colback=blue!0!white,colframe=red!15!white,title=\raisebox{-0.5mm}{\textsc{Why we care:  it affects many people}}]
			\begin{centering}
			\noindent \vspace*{0.5mm}In ageing populations, hearing impairment is a problem affecting the quality of life \small{[2]}.\\[2.7mm]
			%\footnotesize{Hearing Impairment statistics VS age, UK~{[2]}.}\\[1.95mm] %1989
			\hspace{4mm}\includegraphics[width=4.0cm, height=5.10cm, trim=11cm 0cm 0cm 0cm, clip=true]{\SET Demographics_as_per_Davis(1).jpg}\\
			\begin{tikzpicture}[remember picture,overlay] % 
			\foreach \y/\te in {5.05/<30,4.17/31-40,3.29/41-50,2.41/51-60,1.53/61-70,0.65/71-80}{
				\node at (-2.0,\y) [anchor=east] {$\te$};
			}
			\end{tikzpicture}\\[-1mm]
			\end{centering}
			\noindent $\cdot$ The major problem for the hearing impaired or cochlear implantees is communicating in difficult environments.\\[2mm]
			\noindent $\cdot$ We do not understand how the brain segregates and groups sounds, nor how to restore normal hearing. 
		\end{tcolorbox} 	
	\end{minipage}%
\end{posterbox}




%%%%%%%%%%%%%%%%%%%%%%%%%%%%%%%%%%%%%%%%%%%%%%%%%%%%%%%%%%%%%%%%%%%%%%%%%%%%%%	 
\begin{posterbox}[name=sound,column=0, span=1,row=0]{\counterIncrem Sound: \small{A mix of individual objects}}
	\noindent% We hear a mixture of \textbf{sounds} from individual objects.\\
	%\noindent\textbf{Sound}: A propagating wave over 1 second with 3 speakers.\\
	\hspace*{-0.7mm}\textbf{Sounds}\;entering\,our\,ears\,are\,mixtures;\,it's\,not\,colour-coded!\\ 
	\includegraphics[height=2.5cm,width=\cwidth,trim=4.2cm 1.45cm 5.2cm 1cm,clip=true]{\cake}\\
	%\textbf{Spectrogram}: \small{Sound-energy at different frequencies over time} \\[2mm] 
	\hspace*{-0.7mm}Different speakers often use the same range of \textbf{frequencies}. \\[4mm] 
	\includegraphics[height=3cm,width=\cwidth,trim=4.2cm 1.0cm 2.9cm 1.0cm,clip=true]{\specg}\\
	\noindent\begin{tikzpicture}[remember picture,overlay] % 
	 	\coordinate (x) at (8.95,0.46); 
		\coordinate (y) at (9.1,0.46);
		\coordinate (z1) at (0,3.4);
		\coordinate (z2) at (0,3.65);
		\draw[->] (x) -- (y) node[anchor=north east,node distance=0.4cm] {\raisebox{-2mm}{\tiny{time}}};
		\draw[->] (z1) -- (z2) node[anchor=west] {\,\tiny{freq}};
		\node[] at (4.5,3.65) (title) {\tiny{Sound-energy at different frequencies over time}};
		Sound-energy at different frequencies over time
	\end{tikzpicture}\\[-3mm]%
	\noindent \hspace*{-0.7mm}Still  we can \textbf{perceive} the different objects as being separate.\\[1mm] %0.7mm before...
	\foreach \face/\textSaid in {
		% Full heads
		%Iuliz1_2_2_3.jpeg/C'\'etait pour voir si internet,img_matti5_2.jpeg/The cake is a lie,img_alban4_2_lunapic.jpeg/Quoiiiiiiiiiiiiiii}{ 
		% Cut heads
		Iuliz1_2_2_4.jpeg/C'\'etait pour voir si internet$\dots$,img_matti6.jpeg/The cake is a lie,img_alban5.jpeg/Quoiiiiiiiiiii$\dots$ \\[3mm](Whaaaaaa$\dots$)}{
		%\\[3mm](It was to see if internet$\dots$)
		\begin{minipage}{1.05cm}	
			\edef\cimage{\SET guineaPig/\face}
			\includegraphics[height=1.5cm]{\cimage}
		\end{minipage}
		\begin{minipage}{1.8cm}	
			\tiny{\textSaid}
		\end{minipage}%
	}%\\[2mm]
\end{posterbox}

	
%%%%%%%%%%%%%%%%%%%%%%%%%%%%%%%%%%%%%%%%%%%%%%%%%%%%%%%%%%%%%%%%%%%%%%%%%%%%%%	 
\begin{posterbox}[name=ear,column=0, span=1,below=sound]{\counterIncrem Ear \& Cochlear Implant}
	\begin{minipage}{0.8cm}
		\hspace{0.8cm}
	\end{minipage}
	\begin{minipage}{9cm}
		\includegraphics[width=7cm,trim=0cm 0cm 0cm 0.2cm, clip=true]{\cochImp}		 %8.6cm fit
		% Add ref on image
		\begin{tikzpicture}[remember picture,overlay] 
			\node at (0.7,0) [anchor=south east] {[1]};
		\end{tikzpicture}
	\end{minipage}%
\end{posterbox}


%%%%%%%%%%%%%%%%%%%%%%%%%%%%%%%%%%%%%%%%%%%%%%%%%%%%%%%%%%%%%%%%%%%%%%%%%%%%%%	 
\begin{posterbox}[name=soundToSense,column=0, below=ear]{\counterIncrem Cochlea: \small{Waveform to Activity}}
	\noindent
	\textbf{Cochleogram}: \small{Firing of different auditory nerves over time, carrying information about different frequencies~\small{[3]}. 
		Around 50000 nerve fibres transmit sound information to the brain.}\\[2mm] 
		\includegraphics[width=\cwidth,trim=4.2cm 1.0cm 2.9cm 1.0cm,clip=true]{\cochg}\\
	\textbf{Electrode array}: Sequence of activation of a cochlear implant. 
		With a cochlear implant, the information is a fraction ($\sim$1\%) of normal. Still, speech can be understood in quiet environment.\\[2mm]
	\begin{minipage}{\cwidth}
		\begin{tikzpicture}	\begin{scope}[xscale=0.60, yscale=0.60,transform shape] 
			\input{\SET tikzElectrodes/colorsElec_V2.tex}	%\input{\coloElecSix}% \coloElecFift   colorsElec_V2_drawEllipses.tex
		\end{scope}\end{tikzpicture}			
		\begin{tikzpicture}[remember picture,overlay] % 
			\definecolor{lColor}{cmyk}{0.606765,0.685752,0.000000,0.470800}
			\node[anchor=north] at (0.48,0.37) (alpha) {\tiny{$\alpha$}};
			\node[anchor=north] at (0.7525,0.37) (beta) {\tiny{$\beta$}};
			\node[anchor=north] at (1.025,0.37) (gamma) {\tiny{$\gamma$}};
			\node[] at (1.45,-1.64) (alpha2) {\tiny{$\alpha$}};
			\node[] at (4.465,-1.64) (beta2) {\tiny{$\beta$}};
			\node[] at (7.48,-1.64) (gamma2) {\tiny{$\gamma$}};
			%\draw[->,color=lColor] (0.48,    0.25) -- (0.48,0.38);
			%\draw[->,color=lColor] (0.7525,0.25) -- (0.7525,0.38);
			%\draw[->,color=lColor] (1.025,  0.25) -- (1.025,0.38);
		\end{tikzpicture}%
		Every vertical electrode represents the electrical activity of the cochlear implant at a given moment, encoding input sounds:\\[2mm]
		\noindent \hspace{-6mm}
		\begin{tabular}{ccc}
			  \includegraphics[width=2.59cm,trim=1.3cm 2cm 2.4cm 2.2cm,clip=true]{\SET soundPlots/electrodeCochleo_4thElec_time0060000s.pdf}
			&\includegraphics[width=2.59cm,trim=1.3cm 2cm 2.4cm 2.2cm,clip=true]{\SET soundPlots/electrodeCochleo_6thElec_time0090000s.pdf}
			&\includegraphics[width=2.59cm,trim=1.3cm 2cm 2.4cm 2.2cm,clip=true]{\SET soundPlots/electrodeCochleo_8thElec_time0120000s.pdf}\\[0.5mm]
			   time = 0.06s
			& time = 0.09s
			& time = 0.12s
		\end{tabular}
	\end{minipage}
\end{posterbox}


%%%%%%%%%%%%%%%%%%%%%%%%%%%%%%%%%%%%%%%%%%%%%%%%%%%%%%%%%%%%%%%%%%%%%%%%%%%%%%	 
\begin{posterbox}[name=auditoryPathway,column=0, below=soundToSense]{\counterIncrem Auditory Pathway: \small{Sound to Sense}}
	\noindent\includegraphics[width=9.05cm, trim=26mm 15mm 33mm 05mm, clip=true]{\SET Pic_brainstem_heads_3_1.jpg}\\[1mm]
	 Spikes from the auditory nerve pass into the brain where they are processed in a number of discrete brain nuclei. 
	 Somehow, this pathway groups and segregates sounds~\small{[4]}.
\end{posterbox}



%%%%%%%%%%%%%%%%%%%%%%%%%%%%%%%%%%%%%%%%%%%%%%%%%%%%%%%%%%%%%%%%%%%%%%%%%%%%%%	 
%% SECOND COLUMN
%%%%%%%%%%%%%%%%%%%%%%%%%%%%%%%%%%%%%%%%%%%%%%%%%%%%%%%%%%%%%%%%%%%%%%%%%%%%%%	 

\def\cuavePlots{/Users/pmaal/Documents/MATLAB/InGreece/GMM-HMM/htk/CUAVE/plots/}
\edef\soundsPlot{\SET soundPlots/}
\edef\cochleo{\cuavePlots s01m_1_cochleo_20th57Channel.pdf}
\edef\waveform{\SET waveformVcv_pureblue.pdf} 
\edef\waveformBlack{\SET waveformVcv_black.pdf}
\edef\cochleo{\soundsPlot M2ABA7M_1_cochleo_15th57Channel.pdf}	
\edef	\probaFiring{\soundsPlot M2ABA7M_1_15th57thChannel.pdf}
\edef\guineaPig{\SET guineaPig/GP9_light_2_3.png}
\edef\simulatedRaster{\soundsPlot M2ABA7M_1_15th57thChannel_raster.pdf} 
\edef\realRaster{\SET spikeTrainVcv.pdf}
\edef\radio{\SET guineaPig/radio4_2.jpg}
\edef\gianni{\SET guineaPig/Giannis5_3.jpg}
\edef\ABA{\SET guineaPig/ABA2_2.png}
\def\bP{+}%{$\oplus$\;}
\def\bPbold{$\boldsymbol{\oplus}$}
\def\bPbold{$\boldsymbol{+}$}
\def\bM{-}%{$\ominus$\;}
\def\bB{$\bullet$\;}
\newcommand\bText[1]{$\cdot$ #1} 
\newcommand\gianniAbaWaveform[1]{
	\includegraphics[width=1.4cm, trim=0cm 0cm 0cm 0.8cm, clip=true]{\gianni}
	\hspace{-4mm}\raisebox{1.2cm}{\includegraphics[width=0.9cm,angle=-45, trim=0cm 3cm 0cm 0cm, clip=true]{\ABA}}
	\hspace{-0.25cm}\raisebox{0.35cm}{\includegraphics[width=6.5cm, trim=1.9cm 0.75cm 1.4cm 0cm, clip=true]{#1}}	
}
% Column width
\newlength\cwidthRight
\setlength\cwidthRight{15.0cm} 
\newlength\lengCuave
\setlength\lengCuave{8.87cm}

%%%%%%%%%%%%%%%%%%%%%%%%%%% POSTERBOX Modelling VS Experiment
\begin{posterbox}[name=Modelling,column=1, span=2,row=0, below=auditoryNeuro]{\counterIncrem Data: \small{Modelling \& Experiment}}
	\noindent Data can be simulated - using computational models - or obtained from experiments. 
	Each has advantages and limitations:\\[3mm]
	\begin{tabular}{c|c}
		\textbf{Modelling} & 		\textbf{Experiment}\\
		\bText{Speech as input to cochlear model}
		& \bText{Speech played to anaesthetised animal}\\[3mm]
		\gianniAbaWaveform{\waveform}
		&\gianniAbaWaveform{\waveform}\\ %\waveformBlack
		\bText{Algorithm generates neurons' spiking probabilities}
		& \bText{Auditory system generates spikes along Auditory pathway}\\
		\begin{minipage}{\lengCuave}
			\includegraphics[width=\lengCuave, trim=0cm 0.51cm 0cm 0cm, clip=true]{\cochleo}\\
			\includegraphics[width=\lengCuave, trim=0cm 0.51cm 0cm 0cm, clip=true]{\probaFiring}
		\end{minipage}
		& \begin{minipage}{\lengCuave}
			\raisebox{0.0cm}{\includegraphics[width=2.7cm]{\radio}}\begin{minipage}{5cm}\hspace{5cm}\end{minipage}\\[-2cm]
			\begin{minipage}{2cm}\hspace{2cm}\end{minipage}{\includegraphics[width=6cm]{\guineaPig}}
		\end{minipage}\\
		\bText{We generate spikes from firing probability}
		& \bText{We record spiking IC activity from electrodes}\\[2mm]
		\includegraphics[width=\lengCuave, trim=0cm 0cm 0cm 0cm, clip=true]{\simulatedRaster}
		&	\includegraphics[width=\lengCuave, trim=0cm 0cm 0cm 0cm, clip=true]{\realRaster}
		\begin{tikzpicture}[remember picture,overlay]
			\draw[ultra thick,->,red] (-1.35,7.1)  .. controls (0,8) and (1.075,1.0) .. (-0.8,1.0); 
			\definecolor{greenelectrode}{RGB}{218,254,101} % for GP pic 9_2
			%\definecolor{greenelectrode}{cmyk}{1, 0, 1, 0} % for GP pic 9
			%\definecolor{greenelectrode}{RGB}{200,255,177} 
			%\draw[ultra thick,->,color=greenelectrode] (-1.05,6.7)  .. controls (-0.2,7.5) and (0.98,1.5) .. (-0.8,1.5); % single color
			\node[anchor=north west] at (-1.075,6.95) {\includegraphics[width=1.28cm]{\SET guineaPig/guineaPigGreenArrow.png}};
		\end{tikzpicture}\\
		\setlength\tabcolsep{0.075cm}
		\begin{tabular}{cl}
			\bP & Data can be simulated anytime\\
			\bP & Can be computed in big quantities\\ 
			\bP & Parameters can be varied \\
			\bP & Intermediate steps are interpretable \\
			\bM & Each model is based on assumptions
		\end{tabular}
		& 
		\setlength\tabcolsep{0.075cm}
		\begin{tabular}{cl}
			\bPbold & \textbf{This} is the data we ultimately want\\
			\bM&Limitations from legislation (3 R's) [5]\\
			\bM&Experiments are slow and costly \\
			\bM&Cannot predict the quality of data \\
			\bM&Requires training and licenses
		\end{tabular}
	\end{tabular} \\[3.3mm]
	\noindent \textbf{One Sentence Summary}: We develop and test computational and statistical models to improve artificial recognition 
	of sounds in complex scenes from auditory neuronal activity, 
	 in order to probe how the brain itself separates sound sources.\\[-1.65mm] 
\end{posterbox}

%%%%%%%%%%%%%%%%%%%%%%%%%%% POSTERBOX Challenge: \small{Spike train classification}
\begin{posterbox}[name=maths,column=1, span=1, below=Modelling]{\counterIncrem Challenge: \small{Spike train classification}}
\footnotesize
\noindent\hspace{0.35mm} We present here a theoretically sound way to compare spike trains.\\[2.55mm]
\noindent\hspace{-8.5mm}
\raisebox{6mm}{
	\begin{minipage}{4.15cm}
	We represent  a given  spike train  $T=[t_1, \dots, t_f]$ as a function from the Hilbert space $L^2(\mathbb{R})$:\\%[-1mm]
	$$\kappa(T,t):= f\ast T (t) $$\vspace{-3mm}
	$$= \sum_{k=1}^{f}\exp\left(-\frac{(t-t_k)^2}{\sigma^2}\right)$$\vspace{-6mm}
	\end{minipage}
}%
\begin{minipage}{0.0cm}
	\hspace{0.0cm}
\end{minipage}%
\begin{minipage}{4.5cm}
\vspace{-5mm}
% Plot convolved Gaussian with spike trains
% Based on http://tex.stackexchange.com/questions/74574/easiest-way-to-plot-a-function-with-pgf-tikz
\newcommand*{\A}{2}
\pgfmathdeclarefunction{SolutionX}{1}{%
    \pgfmathparse{\t}%
}
\pgfmathdeclarefunction{SolutionY}{2}{%
    \pgfmathparse{exp(-5*(#1-\A)^2)/2+#2}%
}
\pgfmathdeclarefunction{verticalBar}{2}{%
    \pgfmathparse{#1+#2}%
}
\pgfmathdeclarefunction{sumGaussians}{6}{%
    \pgfmathparse{#6+exp(-5*(#1-#2)^2)/2+exp(-5*(#1-#3)^2)/2+exp(-5*(#1-#4)^2)/2+exp(-5*(#1-#5)^2)/2}%
}
% Define styling
\tikzset{My Line Style/.style={smooth, ultra thick, color=red, samples=400}}
\tikzset{My Vertical Bar/.style={smooth, ultra thick, color=black, samples=400}}
\tikzset{My Gaussian Style/.style={smooth, color=gray, dashed, thick, samples=400}}	
\begin{tikzpicture}[x=95cm,y=1.6cm] 
	% Make axis same scale as tikzpicture (x= and y=)
	\begin{axis}[ anchor=origin,axis lines=middle, xmin=0, xmax = 5, ymin=-1.25, ymax = 0.75,x=0.95cm,y=1.6cm, hide axis, xtick=\empty, ytick=\empty]
		% Labels below spikes or top right of curve
		\node at (axis cs:1,-0.03) [anchor=north ] {$t_1$};
		\node at (axis cs:2,-0.03) [anchor=north ] {$t_2$};
		\node at (axis cs:2.5,-0.03) [anchor=north] {$t_3$};
		\node at (axis cs:4,-0.03) [anchor=north ] {$t_4$};
		\node at (axis cs:4.5,0.56) [color=red] {$ f\ast T $};
		% x-axis
		\addplot[variable=\t, domain=0:5] ({\t}, {0});							
		\addplot[variable=\t, domain=0:5] ({\t}, {-1});		
		% Gaussians and vertical dots					
		\foreach \ti/\ind in {1/1, 2/2, 2.5/3, 4/4}{
			\edef\A{\ti}
   			\addplot[My Gaussian Style,  variable=\t, domain=0:5]    ({SolutionX(\t)},{SolutionY(\t,0)});		
			\addplot[densely dotted, variable=\t, domain=0:-1] ({\ti}, {\t});
		}
		% Vertical bars
		\foreach \ti/\ind in {1/1, 2/2, 2.5/3, 4/4}{
   			\addplot[My Vertical Bar,        variable=\t, domain=0:0.25]   ({\ti},{verticalBar(\t,0)});	
		}
		% Labels below spikes  or top right of curve
		\node at (axis cs:1.1,-1.03) [anchor=north ] {$\tilde{t}_1$};
		\node at (axis cs:1.8,-1.03) [anchor=north ] {$\tilde{t}_2$};
		\node at (axis cs:2.6,-1.03) [anchor=north ] {$\tilde{t}_3$};
		\node at (axis cs:3.6,-1.03) [anchor=north ] {$\tilde{t}_4$};
		\node at (axis cs:4.5,-0.76) [color=red] {$ f\ast \tilde{T} $};
		% Sum Gaussians
		\addplot[My Line Style,  variable=\t, domain=0:5]    ({\t},{sumGaussians(\t,1, 2, 2.5, 4,0)});	
		\foreach \ti in {1.1, 1.8, 2.6, 3.6}{
			\edef\A{\ti}
   			\addplot[My Gaussian Style,  variable=\t, domain=0:5]    ({SolutionX(\t)},{SolutionY(\t,-1)});		
		}
		% Vertical Bars
		\foreach \ti in {1.1, 1.8, 2.6, 3.6}{
   			\addplot[My Vertical Bar,        variable=\t, domain=0:0.25]   ({\ti},{verticalBar(\t,-1)});		
		}
		% Sum Gaussians
		\addplot[My Line Style,  variable=\t, domain=0:5]    ({\t},{sumGaussians(\t,1.1, 1.8, 2.6, 3.6,-1)});									
	\end{axis}
\end{tikzpicture}
\end{minipage}\\
\begin{minipage}{9.0cm}
where 
$f(t) = \text{e}^{-t^2/\sigma^2}$
 is the kernel we convolve spike trains with. This `\textbf{Kernel Trick}' allows one to perform signal processing on neural data, as shown in this example of classification: From two sets of spike trains \textcolor{red}{$T_1 \dots$} and \textcolor{blue}{$\tilde{T}_1\dots$ }coming from an experiment with two different setups, how can we build a model to guess the setup associated with a new spike trains?
 \end{minipage}\\[3mm]
\newlength\lengCla
\setlength\lengCla{3cm}
\hspace{-7mm}
\begin{tabular}{cc}
	\begin{minipage}{\lengCla}
		\includegraphics[width=\lengCla,trim=2cm 1.5cm 1cm 1cm,clip=true]{\xy}  
	\end{minipage}  
&	\hspace{-16.5mm}\begin{minipage}{0.8\lengCla}
		\foreach \i in {1,2}{%
			{$\textcolor{red}{y_\i:=%\kappa(T_\i, \cdot) = 
			f\ast T_\i}$}\\
		}
		\vspace{0mm}\hspace{5mm}\textcolor{red}{\vdots}\\
		\noindent\foreach \i in {1,2}{%
			{$\textcolor{blue}{x_\i:=%\kappa(\tilde{T}_\i, \cdot) = 
			f\ast \tilde{T}_\i}$}\\
		}
		\vspace{0mm}\hspace{5mm}\textcolor{blue}{\vdots}
	\end{minipage}
	\begin{minipage}{0.3\lengCla}
		\vspace{-0.2cm}	\textcolor{red}{$$y_{Mean} = \frac{1}{M}\sum_{j=1}^{M}y_j$$}
		\vspace{0.0cm}		\textcolor{blue}{$$x_{Mean} = \frac{1}{N}\sum_{i=1}^{N}x_i$$ }
	\end{minipage}\\[5mm] % up to 2
	\begin{minipage}{\lengCla}
		\includegraphics[width=\lengCla,trim=2cm 1.5cm 1cm 1cm,clip=true]{\xyMSC}  
	\end{minipage}  
&	\hspace{-3.5mm}\begin{minipage}{5.5cm}
		\textbf{Classification}: For 
		$z = [t_1, t_2 \cdots, t_K]$ a new spike train, we colour it depending on the sign of
		$$ ||y_{Mean}-\kappa(z, \cdot)|| - || x_{Mean}-\kappa(z, \cdot)||$$ 
	\end{minipage} 
\end{tabular}
\end{posterbox}



% name=auditoryNeuro,column=1, span=2,row=0,headerColorOne=red!40!black,headerColorTwo=red!75!white,borderColor=red!50!white, textborder=roundedsmall
%%%%%%%%%%%%%%%%%%%%%%%%%%% POSTERBOX
\begin{posterbox}[name=Conclusion,column=2, span=1, below=Modelling,headerColorOne=red!40!black,headerColorTwo=red!75!white,borderColor=red!50!white]{Conclusion}
\large
\noindent Hearing is an interdisciplinary problem related to Big Data/Data Science challenges.\;%\\[2mm]
The development of new technologies, computational models and statistical techniques 
will be necessary to solve its mysteries and improve hearing aids. %\\
\end{posterbox}



%%%%%%%%%%%%%%%%%%%%%%%%%%% POSTERBOX
\begin{posterbox}[name=References,column=2, span=1, below=Conclusion]{References} 
\footnotesize{%
\noindent [1] Blausen gallery 2014, Wikiversity Journal of Medicine.\\% DOI:10.15347/wjm/2014.010. ISSN 20018762.\\
%https://commons.wikimedia.org/wiki/File:\\Blausen\_0244\_CochlearImplant\_01.png\\%
\noindent [2] Davis AC, International Journal of Epidemiology (1989). \\%
\noindent [3] Sumner CJ, Lopez-Poveda EA, O'Mard LP \& Meddis R (2002).\\%
\noindent [4] Plack C, The sense of hearing (2005).\\%
\noindent [5] Russell WMS \& Burch RL, The Principles of Humane Experimental Technique  (1959).\\
\noindent [BG] Fotis Bobolas, Neurons (2009).
}
\end{posterbox}

%%%%%%%%%%%%%%%%%%%%%%%%%%% POSTERBOX
\begin{posterbox}[name=Acknowledgment,column=2, span=1, below=References]{Acknowledgment}
\normalsize
\noindent \begin{tikzpicture}[remember picture,overlay]
	\node[] at (2.05,-2.40) {\textcolor{colorTitle}{\includegraphics[width=0.75cm]{\SET flags/UKflag_2.png}}}; 
	\node[] at (5.05,-2.65) {\textcolor{colorTitle}{\texttt{Alban.Levy@nottingham.ac.uk}}}; 
	\node[] at (5.05,-2.15) {\textcolor{colorTitle}{SET for Britain, $7^{\text{th}}$ March 2016}}; 
\end{tikzpicture}%
\footnotesize
\noindent Support from European Commission Grant\#289146, 
	University of Nottingham:\,%Nottingham University: 
	MRC Institute of Hearing Research${}^\characB$ 
	School of Mathematical Sciences${}^\characA$, 
	Athens Information Technology${}^\characC$,
	David Hawker, Ian Dryden,  Florian Levity, Sid Visser, the guinea-pigs.
\end{posterbox}


%%%%%%%%%%%%%%%%%%%%%%%%%%% END

\end{poster}

\end{document}