\documentclass[]{beamer} % handout: for removing the \pause
\usepackage[french,british]{babel}
\usetheme[compress]{Singapore} % Compress: horizontal bullets of subsections
\setbeamertemplate{navigation symbols}{}  % Get rid of navigation symbols

\usefonttheme{serif}
\usepackage[T1]{fontenc}
\usepackage{graphics}
\usepackage{verbatim}
\usepackage{minitoc}
\usepackage[pages=some]{background}
\backgroundsetup{
    pages=some,
    placement=top,
    angle=45,
    scale=1,
    contents=\raisebox{-1.5cm}{ADVANCED},
    opacity=0.5
}
\usepackage{graphicx}
\usepackage{xcolor}
\usepackage{hyperref}
\definecolor{darkblue}{rgb}{0.0, 0.0, 0.7}
\hypersetup{
     colorlinks   = true,
     citecolor    = darkblue,
     linkcolor    = darkblue,
     filecolor    = darkblue,
     urlcolor     = darkblue 
}     


% PATHS
\IfFileExists{/Users/pmxal9/Dropbox/Logos/RxPz.jpg}{
	\def\pathbase{/Users/pmxal9/} 	% Lab computer
}{	
	\def\pathbase{/Users/pmaal/} 	% Personal computer
}
\edef\drop{\pathbase Dropbox/}		% Dropbox: shared between computers
\edef\compPhdGeneral{\drop Nottingham/repos/phd/Seminars/ComputationalPhD/}
\edef\compPhd{\drop Nottingham/repos/phd/Seminars/+MatlabComputationalPhD/}

% LOGOS
\edef\logoRux{\drop Logos/RxPz.jpg}
\edef\logoIHR{\drop Logos/ihr.png}
\edef\logoPerso{\drop Logos/Logo.jpg}
\edef\logoNotts{\drop Logos/notts.png}

% TITLEPAGE
\title[Matlab for your Computational PhD]{\small{}\\
	\small{\textsc{Matlab for your Computational PhD}\\[5mm]}
}
\author[Levity]{
	\raisebox{0.4cm}{}
	\includegraphics[height=1.1cm]{\logoRux}\\[12mm]
	\includegraphics[height=1.0cm]{\logoIHR}\hspace{0.8cm}
	\includegraphics[height=1.2cm]{\logoPerso}  \hspace{0.7cm}
	\includegraphics[height=1.05cm]{\logoNotts}\\[5mm]
	\textcolor{gray}{\small{pmxal9@nottingham.ac.uk}} \vspace{-10mm} 
}
\date{\small{\textcolor{gray}{12 Oct 2016}}}

% TOC
\setcounter{tocdepth}{1} 

%%%%%%% 
% DOCUMENT
\begin{document}


% First frame
\setbeamertemplate{footline}{} %no foot line on this page
\frame[noframenumbering]{\titlepage}

% Footlines: current page only
\setbeamertemplate{footline}{\hfill\raisebox{2mm}[0pt][0pt]{\insertframenumber}\hspace*{2mm}} 
\beamertemplatenavigationsymbolsempty

% TOC
\frame{\tableofcontents}

% Reset Headline style
\setbeamertemplate{headline}
 {%
  \begin{beamercolorbox}{section in head/foot}
  \insertsectionnavigationhorizontal{\textwidth}{}{}
  \end{beamercolorbox}%
}


%%%%%%%%%%%
\section[I]{Introduction}
%%%%%%%%%%%

\subsection{Introduction}
\begin{frame}{This talk}
What this talk is about:
\begin{itemize}
\item Introducing the Matlab environment,
\item Giving tips for a better use of Matlab,
\item Giving pointers you can dig at home,
\item Recommending Software Engineering practices.\\[5mm]
\end{itemize}
\pause
What this talk is not about:
\begin{itemize}
\item Numerical schemes,
\item Modelling,
\item Big Data,
\item Happiness and how is happens.
\end{itemize}
\end{frame}

\newcommand\csize[1]{\small{#1}}

\subsection{Keep in mind}
\begin{frame}{To Keep in Mind During a PhD (self-help++)} 
\csize{\url{https://graduable.com/2012/10/04/21-things-every-phd-student-should-know/}}\\
Take care of yourself, vision of Academia likely far from reality, learn to read, it's okay to change ... anything (project, supervisors), 
depression, get to know the other postgrads, \dots\\[3mm]
\csize{\url{https://www.theguardian.com/higher-education-network/blog/2014/may/02/five-things-successful-phd-students-refuse-to-do}}\\
Feel like a failure, feel out of control...\\[3mm]
\csize{\url{http://www.nextscientist.com/graduate-school-advice-series-starting-phd/}}\\
Aim to publish peer-reviewed article, hard to stay motivated, networking is important\dots\\[3mm]
%\url{https://www.timeshighereducation.com/carousels/essential-phd-tips-10-articles-all-doctoral-students-should-read}\\
Be curious: Look for/after yourself :)
\end{frame}


\subsection{What is a  PhD?}
\begin{frame}{What is a  PhD?}
\begin{itemize}
\item Getting skills to perform research tasks,
\item Become the expert on one topic,
\item Start your career (make a network and publish),
\item Move your research field forward (in practice, that's often long after the PhD, but act as if).
\end{itemize}
So why do good Software Engineering practices matter?\\
\pause
\begin{itemize}
\item Scientists spend > 30\% of their time on coding {\tiny{[1]}},
\item More than 90\% are self-taught {\tiny{[1]}},
\item Substantial quality problems,
\item Increase of scientific errors due to incorrect Software {\tiny{[2]}},
\item Opens up job/career opportunities, particularly in industry.
\end{itemize}
\tiny{[1] P. Prabhu et al, Proc. 24th ACM/IEEE Conference on HPC, Networking, Storage and Analysis (2011)}\\
\tiny{[2] Z. Merali, Error: Why scientific programming does not compute. Nature 467: 775777 (2010).}
\end{frame}


\begin{frame}{What is a  PhD Thesis?}
Umberto Eco takes us back to the original purpose of theses and dissertations as defining events that conclude a program of study. 
They are not a test or an exam, nor should they be. [...] %They are not meant to prove that the student did his or her homework. 
Rather, they prove that students can \textbf{make something out of their education}.
This is particularly important today, when we are more accustomed to thinking in compliance with the software of 
our laptop or doing research according to the logic of a tablet than to thinking and researching in a personal and independent way.\\[2mm]

The humanities are intrinsically creative and innovative. 
They are about originality and invention, not discovery. 
This is precisely Eco's testimony; even more than a technical manual,
this book is an \textbf{invitation to ingenuity, a tribute to imagination}.\\[2mm]

Preface of `\textit{How to Write a Thesis}' 
\end{frame}


\begin{frame}{Supervision}
Or its absence\dots \\
\url{http://academiaiskillingmyfriends.tumblr.com/}\\[5mm]
\pause
Most supervisors consider your programming/SE skills are your problem. This is bound to change, but takes time... For now,
\pause
\begin{center}\textcolor{blue}{It's your responsibility.}\end{center}
\pause 
Partner up and go. \\[3mm]
\pause 
Another pitfall is to spend too much time on it. Research remains your goal.
\end{frame}


\begin{frame}{Best Practices for Scientific Computing}
%Best Practices for Scientific Computing, PLoS Biology 2014. 
%Scientists typically develop their own software for these purposes because doing so requires substantial domain-specific knowledge. 
%As a result, recent studies have found that scientists typically spend 30\% or more of their time developing software. 
%However, 90\% or more of them are primarily self-taught, and therefore 
Scientists [...]  \textbf{lack exposure to basic 
software development practices} such as writing maintainable code, 
using version control and issue trackers, code reviews, unit testing, and task automation.\\[3mm]

A large body of research has shown that \textbf{code reviews are the most cost-effective way of finding bugs in code}. 
They are also a good way to spread knowledge and good practices around a team. In projects with shifting membership, 
such as most academic labs, code reviews help ensure that critical knowledge isn't lost when a student or postdoc leaves the lab.\\[3mm]
\noindent {\tiny{\url{http://journals.plos.org/plosbiology/article?id=10.1371/journal.pbio.1001745}}}
\end{frame}


\begin{frame}{Code Review}\centering
\def\file{xkcd/code_quality}
\includegraphics[width=11cm]{\compPhd \file}
\end{frame}


\begin{frame}{General Tips on your Research (1/3)}
\begin{itemize}
\item Share your ideas, 
\item Let go of what does not work, 
\item Ask for advice after having given some thoughts
\end{itemize}
\def\file{intro/smartpeople}
\begin{center}\includegraphics[width=8cm]{\compPhd \file}\end{center}
{\tiny{\url{http://www.hbs.edu/faculty/Publication\%20Files/Advice\%20Seeking_59ad2c42-54d6-4b32-8517-a99eeae0a45c.pdf}}}
\end{frame}


\begin{frame}[fragile]
\frametitle{General Tips on your Computer (2/3)}
\begin{block}{Use Version Control}
Bitbucket, Git\dots (bitbucket has private folders)
\end{block}
\begin{block}{Bibliography Manager}
Choose one, starting today: Zotero, Mendeley\dots
\end{block}
\begin{block}{Notations}
No space in folders \& file names
\end{block}
\begin{block}{Folder Organisation}
Think carefully of a system \&  stick to it (until next version)\\
\small{\url{http://www.howtogeek.com/howto/15677/zen-and-the-art-of-file-and-folder-organization/}}
\end{block}
\end{frame}


\begin{frame}[fragile]
\frametitle{General Tips on your Computer (3/3)}
\begin{block}{Back-up}
Your computer will crash and you'll lose your data
\end{block}
\begin{block}{LaTeX is a programming language}
Modularity with \verb+\input+, make useful commands with \verb+\newcommand+, make variables with 
\verb+\def+ or \verb+\edef+ for your computer paths\dots
\end{block}
\begin{block}{Command-line Editor} 
Learn to use the servers (Vim, Emacs\dots)
\end{block}
\begin{block}{Unix}
Learn Unix basics: \verb+cat, grep, sed, find, ssh, scp, man, head, cd,+\\
\verb+ls, rm, wc, xargs+.\\
\small{\url{http://swcarpentry.github.io/shell-novice/})}
\end{block}
\end{frame}


\begin{frame}
\frametitle{General Tips on your Computer (4/3)}
\begin{block}{Back-up}
Buy a hard-drive, online (Dropbox,  Bitbucket) or whatever.\\
Like, seriously...
\end{block}
\end{frame}


%%%%%%%%%%%%%%%%%%%%
\section[M]{Matlab}
%%%%%%%%%%%%%%%%%%%%
%\frame{\tableofcontents[currentsection]} %,sectionstyle=show/hide,subsectionstyle=show/show/hide]}


\begin{frame}{Choosing Your Language}\centering
\fcolorbox{black}{white}{
\begin{minipage}{9cm}
\includegraphics[width=9cm]{\compPhd mariomulansky/nomatlab1.png}\\
\pause
\includegraphics[width=9cm]{\compPhd mariomulansky/nomatlab2.png}
\end{minipage}
}\\[3mm]
\hfill Mario Mulansky, ex-NETT,  currently working for Apple
\end{frame}


\newcommand*\cg{\tikz[baseline=(char.base)]{\raisebox{0.12cm}{\node[circle,fill=green,draw,inner sep=2pt,opacity=0.5,text opacity=1] (char) {};}}}
\newcommand*\ccr{\tikz[baseline=(char.base)]{\raisebox{0.12cm}{\node[circle,fill=red,draw,inner sep=2pt,opacity=0.5,text opacity=1] (char) {};}}}


\begin{frame}{Matlab}
Don't use Matlab... but if you have to, know your tool! \\
Matlab...
\begin{itemize}[<+(1)->]
\item[\cg] is a nice working environment,
\item[\cg] has a lot of useful toolboxes (and we have many: run `ver')
\item[\cg] is great for fast prototyping,
\item[\cg] has OOP capabilities since 2008,
\item[\cg] has some computational strengths,
\item[\cg] is used in many Science fields and in industry,
\item[\ccr] is expensive (alternative: Octave),
\item[\ccr] has memory leaks,
\item[\ccr] is weak for text processing,
\item[\ccr] is easily slow,
\item[\ccr] ... 
\end{itemize}
\end{frame}


\subsection{How to use Matlab}
\begin{frame}{How to use Matlab}\centering
\begin{itemize}[<+(0)->]
\item Command-line in Matlab environment\\
\includegraphics[width=4cm]{\compPhd howToUseMatlab/commandwindow.png}
\item Running scripts in Matlab command window\\
\includegraphics[width=9.2cm]{\compPhd howToUseMatlab/script2.png}
\item Command-line in a shell (fastest)\\
\$ matlab -nodisplay -nosplash < path/to/myscript.m
%/Users/pmaal/Dropbox/Nottingham/repos/phd/Seminars/+MatlabComputationalPhD/howToUseMatlab/myscript.m 
\item Write GUIs
\end{itemize}
\end{frame}


\subsection{Adapt your Matlab Environment}
\begin{frame}{Adapt your Matlab Environment}\centering
\begin{block}{Home > Layout}
\pause
\begin{itemize}
\item Add column,
\item Move layout here and there,
\item Add/Remove Workspace, Command History,
\item ...
\end{itemize}
\end{block}
\end{frame}


\subsection{Functions VS Scripts}
\begin{frame}[fragile]%
\centering
\frametitle{Functions VS Scripts}
\begin{block}{\textbf{Functions} should do one thing only}
and do it well. Their name should give this information, a help describe their use and incorporate a minimal working example.
\end{block}
\begin{exampleblock}{Example}
\vspace{-4mm}
\begin{verbatim}cellText = importdata('myfile.txt'); 
firstValue = str2double(cellText{1});
\end{verbatim}
\end{exampleblock}
\end{frame}


\begin{frame}[fragile]%
\centering
\frametitle{Functions VS Scripts}
\begin{block}{\textbf{Scripts} should be your workflow }
and clearly show what you do along the way.
\end{block}
\begin{exampleblock}{Example}
\vspace{-4mm}
\begin{verbatim}a = randn(100); 
eigA = eig(a); 
figure; plot(eigA, '.');
saveas(gcf, 'path/to/eigA.pdf');
\end{verbatim}
\end{exampleblock}
\end{frame}


\subsection{Base and Function Workspaces}
\begin{frame}{Workspaces}%\centering
All the variables that are accessible within a function are
\begin{itemize}
\item either given as inputs to the function, or
\item loaded or computed within the function.\\[3mm]
\end{itemize}
\begin{exampleblock}{Example}%
{Within the function song2txt, }%
just after the first line of code is executed, only the variables wavfile, txtfile, wav and fs 
exist! \\[2mm]
\begin{center}\includegraphics[width=9.2cm]{\compPhd workspaces/song2txt.png}\end{center}
%\url{https://uk.mathworks.com/help/matlab/matlab\_prog/base-and-function-workspaces.html}
\end{exampleblock}
\end{frame}


\begin{frame}[fragile]%
\frametitle{Workspaces}
The \textbf{Base workspace} stores variables that you create at the command line. 
This includes any variables that scripts create, assuming that you run the script from the command line or from the Editor.\\[3mm]

Functions do not use the base workspace. Every function has its own \textbf{Function workspace}. 
Each function workspace is separate from the base workspace and all other workspaces to protect the integrity of the data.
\pause
\begin{exampleblock}{Example}%
\begin{center}\includegraphics[width=9.6cm]{\compPhd workspaces/scriptfunciton.png}\end{center}
What is the value of `a' after we run 
\verb+a = 0; myworkscript; myworkfunction();+? 
\end{exampleblock}
\end{frame}


\subsection{Sharing Data across Workspaces}
\begin{frame}{Sharing Data across Workspaces}
In most cases, variables created within a function are local variables known only within that function. 
But, since sharing is caring\dots\\
\begin{itemize}
\item Passing arguments (best practice),
\item Nested functions (not very popular),
\item Evaluating in another workspace (bad practice),
\item Saving data in a file and loading it later (slow),
\item Persistent variables (very useful),
\item Global variables (bad practice). \\[5mm]
\end{itemize}
\small{\url{https://uk.mathworks.com/help/matlab/matlab_prog/share-data-between-workspaces.html}}
\end{frame}


\foreach \file in {nestedfunctions, evalinassignin, persistentvariables, globalvariables} {
	\begin{frame}{Sharing Data across Workspaces}
	\begin{center}\includegraphics[width=9.6cm]{\compPhd workspaces/\file}\end{center}
	\end{frame}
}


\begin{frame}{Sharing Data across Workspaces}
A common way of passing many arguments with only a few inputs is to use structures.
\def\file{myworkfunction2}
\begin{exampleblock}{Example: \file}%
\begin{center}\includegraphics[width=9.6cm]{\compPhd workspaces/\file}\end{center}
\end{exampleblock}
\end{frame}


\begin{frame}{Sharing Data across Workspaces}
Let's compare with two other possibilities\\
\hspace{-05mm}%
\foreach \file in {myworkfunction1, myworkfunction3}{%
	\begin{minipage}{5.8cm}
	\begin{exampleblock}{Example: \file}%
	\begin{center}\includegraphics[width=5.8cm]{\compPhd workspaces/\file}\end{center}
	\end{exampleblock}%
	\end{minipage}%
}%
\end{frame}


\subsection{Stack}
\begin{frame}[fragile]
\frametitle{Stack}\centering
\begin{block}{Stack}%
When calling functions within functions, Matlab keeps track of the order in which functions call other functions in the \textbf{stack}.
In debug mode, \verb+cDb = dbstack();+ outputs a structure array. 
You can also navigate it using the `Function Call Stack'.
\end{block}
\def\file{functioncallstack}
\begin{exampleblock}{Example}%
	\begin{center}\includegraphics[width=10.0cm]{\compPhd workspaces/\file}\end{center}
\end{exampleblock}
\end{frame}


\subsection{Debugging}
\begin{frame}[fragile]
\frametitle{Debugging}
Debugging takes time: help yourself by reducing it. 
\end{frame}


\begin{frame}[fragile]
\frametitle{Debugging}
\def\file{googletime}
\begin{center}\includegraphics[width=10.0cm]{\compPhd debugging/\file}\end{center}
\end{frame}


\begin{frame}[fragile]
\frametitle{Debugging}
Enter Debugging mode by setting Breakpoints within your code, so that you can access and check all values at this point. 
\def\file{debugclick}
\begin{center}\includegraphics[width=10.0cm]{\compPhd debugging/\file}\end{center}
\end{frame}


\begin{frame}[fragile]
\frametitle{Debugging}
In debug mode, you can see the function variables within the `Worspace' tab, and read variables values by hovering over them.\\%
\def\file{debug2}%
\begin{center}\hspace{0mm}\includegraphics[width=9.0cm]{\compPhd debugging/\file}\end{center}
\end{frame}


\begin{frame}[fragile]
\frametitle{Debugging}
\begin{block}{Debugging tip}
Use \verb+dbstop if error+ during development to enter debug mode instead of crashing. 
Don't use in general because most of the time you know what the error is by reading the error message. \\[3mm]
Disable with \verb+dbclear if error+.
\end{block}
\end{frame}


\subsection{Data Types}
\begin{frame}{Numeric Types  }\centering
\begin{block}{Numeric Types}
Double, single, int8, int16, int32, int64,  uint8, \dots\\[2mm]
%adou = double(1); asin = single(1); aint8 = int8(1); aint16 = int16(1); auint16 = uint16(1);
\includegraphics[width=6.0cm]{\compPhd dataTypes/data.png}
\end{block}
Big Data? Bottlenecks? Depending on your needs, some numeric types could  be better suited than others.
\end{frame}


\begin{frame}[fragile]
\frametitle{Main Data Classes}
\begin{tabular}{l|l}
Array & \verb+a = [1, -2.5]; b = randn(3); c = 'hi';+\\
Cell & \verb+a = {1, [1 2], {1 2 3}, 'hi'};+\\
Structure & \verb+a.f1 = 1; a.f2 = '2'; a.f3 = {4};+\\
Struct. array & \verb+a(2).f1 = 0;+\\
Table  &  \verb+a = array2table([1 2],'Var',{'x' 'y'})+\\
Categorical & \verb+a = categorical([3 3 4 4])+\\ 
function\_handle & \verb+a = @(x)2*x+ \\ 
(OOP) & \verb+a = albanissocurly;+ \\
(Graphical) & \verb+a = figure;+\\[3mm]
 \end{tabular}
Functions \verb+class+, \verb+size+ and \verb+isa+ useful for automation.
\end{frame}


\begin{frame}[fragile]
\frametitle{Structure Array VS Tables}\centering
Structure arrays are more common among Matlab users than Tables, but the latter present a certain number of advantages:\\[4mm]
\begin{tabular}{c | c|c}
& Stuct. Array & Table\\ \hline
\begin{minipage}{3.8cm} 
Extracting column of numbers/strings~takes 
\end{minipage}
&
\begin{minipage}{2.8cm}  \vspace{1mm}two semantics: \\
\verb+a1 = [a.dou]; +\\
\verb+a2 = {a.str};+\vspace{1mm}\end{minipage}
& 
\begin{minipage}{2.8cm} \vspace{1mm}one semantic:\\
\verb+a1 = a.dou; +\\
\verb+a2 = a.str;+\vspace{1mm}\end{minipage}\\ \hline
\begin{minipage}{3.8cm} 
Switch to categories to reduce redundancy is\end{minipage}
&
\begin{minipage}{2.8cm} \vspace{1mm}not easy\end{minipage}
& \begin{minipage}{2.8cm} \vspace{1mm}easy: \\
\verb+a.str = +\\
\verb+ categorical(a.str)+\vspace{1mm}
\end{minipage}\\ \hline
\begin{minipage}{3.5cm} 
Having a summary of the information is\end{minipage}
&
\begin{minipage}{2.8cm} not easy\end{minipage}
& \begin{minipage}{2.8cm} \vspace{1mm}easy: \\
\verb+summary(a)+\\
\end{minipage}
\end{tabular}
\end{frame}


\begin{frame}[fragile]
\frametitle{Structure Array VS Tables}\centering
\begin{exampleblock}{Example: >> summary(T)}\vspace{-4mm}
\begin{verbatim}
Description:  Simulated Patient Data
Variables:
    Gender: 100x1 cell string
        Description:  Male or Female
    Age: 100x1 double
        Units:  Yrs
        Values:
            min       25   
            median    39   
            max       50   
    Smoker: 100x1 logical
        Description:  true or false
        Values:
            true     34      
            false    66      
\end{verbatim}
% https://uk.mathworks.com/help/matlab/matlab_prog/advantages-of-using-tables.html
\end{exampleblock}
\end{frame}


\begin{frame}[fragile]
\frametitle{Structure Array VS Tables}\centering
\begin{exampleblock}{Example: Reduce redundancy}
\def\file{dou}
\begin{center}\hspace{0mm}\includegraphics[width=9.0cm]{\compPhd structarraytable/\file}\end{center}
\pause
\def\file{structarraytable/str}
\begin{center}\hspace{0mm}\includegraphics[width=9.0cm]{\compPhd \file}\end{center}
\end{exampleblock}
If your table contains a lot of redundancy, using categories can be very time- and memory-efficient. 
Requires care.
\end{frame}


\begin{frame}[fragile]
\frametitle{Structure Array VS Tables}
\begin{block}{Take-home message}
If you need to create a lot of complex data (Data Wrangling), think of the simplest table you can make to contain it. 
The redundancy this entails can often be  dealt with by using categories.
\end{block}
\begin{exampleblock}{Example}
\ccr\,~\verb+myData(5).meta(19).metaMeta(4).damnThisIsACell{4} = 1;+\\
\cg\, \verb+myTable.ThisIsACell(4) = 1;+\\[3mm]
\end{exampleblock}
\url{https://uk.mathworks.com/help/matlab/ref/table.html}
\end{frame}


\subsection{Data Exploration}
\begin{frame}[fragile]
\frametitle{Data Exploration}
\begin{block}{Run in Matlab Command Window}
\vspace{-3.5mm}
\begin{verbatim}mystruct.data = [linspace(0,20,100)', rand(100,1)];
mystruct.description = 'Description';
open mystruct\end{verbatim}
\end{block}
%
\begin{block}{Click on `data' field}
\end{block}
%
\begin{block}{Click on PLOTS > pie}
\end{block}
%
\begin{block}{Repeat on a column of `data'}
\end{block}
\end{frame}


\subsection{Loading \& Saving Data}
\begin{frame}{Loading \& Saving Data}\centering
Three options for saving: 
\begin{itemize}
\item Save Matlab variables in a .mat file to reuse them later \\
	\cg\, Simplest\\\ccr\, Matlab only
\item Save in a text file\\
	\cg\, Readable output (universal interface)\\
	\cg\, Can be used by all programming languages \\
	\cg\, Can use Bash utilities \\
	\ccr\, Requires some planning ahead: appropriate format?
\item Save in a binary file\\
	\cg\, Minimal size\\
	\ccr\, Takes more programming time\\
	\ccr\, Requires a lot of planning ahead: optimised format?\\
	\ccr\, Non-readable output, hence much harder to debug
\end{itemize}
\end{frame}


\begin{frame}[fragile]
\frametitle{Loading \& Saving Data}
Loading data: 
\begin{itemize}
\item Matlab variables: \verb+load+
\item Audio: \verb+audioread+
\item Images: \verb+imread+
\item Binary file: \verb+fscanf, fgetl, fgets, fread+ 
%https://uk.mathworks.com/help/matlab/import_export/import-text-data-files-with-low-level-io.html
\item Text: \verb+textscan, csvread, dlmread, readtable+
\end{itemize}
The import tool (Home >  Variable > Import Data) and \verb+importdata+ are quite general, but less robust. 
\end{frame}


\begin{frame}[fragile]
\frametitle{Loading \& Saving Data}
\begin{exampleblock}{Use Importdata to import myfile.txt into a structure}
\vspace{-5mm}\begin{verbatim}>> type('myfile.txt')
Day1  Day2  Day3  Day4  
95.01 76.21 61.54 40.57  
23.11 45.65 79.19 93.55 
60.68  1.85 92.18 91.69 
48.60 82.14 73.82 41.03  
89.13 44.47 17.63 89.36 
>> M = importdata('myfile.txt', ' ', 1);
>> disp(M);
M = 

          data: [5x4 double]
      textdata: {'Day1'  'Day2'  'Day3'  'Day4'}
    colheaders: {'Day1'  'Day2'  'Day3'  'Day4'}
    \end{verbatim}
\end{exampleblock}
\end{frame}


\subsection{Saving Plots}
\begin{frame}[fragile]
\frametitle{Saving Plots}\centering
\begin{block}{Automate saving}
Plots are quite important; \\
Freezing how they were made is underrated,\\
Keep .fig to avoid having to recreate it when changing the format.
\end{block}
\begin{exampleblock}{Example}\vspace{-5mm}
\begin{verbatim}set(gcf, 'PaperSize', [12 4], 'PaperPosition', [0 0 12 4]);
soundPlots = sprintf('/Users/exp1058/%s',func2str(func)); 
saveas(gcf, [soundPlots '.pdf']);
saveas(gcf, [soundPlots '.fig']);
\end{verbatim}
\end{exampleblock}
\end{frame}


\subsection{Profiling}
\begin{frame}[fragile]
\frametitle{Profiling}\centering
Profiling to increase performance comes last:\\
\textit{Make it work. Make it right. Make it fast.}\\[8mm]
\pause
\begin{block}{Main Tools}
\begin{itemize}
\item \verb+tic, toc+
\item \verb+profile on, profile viewer, profile off+,
\item Home > Code > Run and Time.
\end{itemize}
\end{block}
\end{frame}


\begin{frame}[fragile]
\frametitle{Profiling}\centering
\def\file{tictoc}
\includegraphics[width=7.0cm]{\compPhd profiling/\file}
\end{frame}


\begin{frame}{Profiling}\centering
\def\file{profileviewer}
\begin{minipage}{4cm}
\includegraphics[height=2.0cm]{\compPhd profiling/profviewer1}\\
or from tictocs.m\\[4mm]
\includegraphics[height=2.0cm]{\compPhd profiling/profviewer2}
\end{minipage}%
\pause\begin{minipage}{6cm}
\includegraphics[height=7.0cm, trim=0cm 0cm 0cm 5.7cm, clip=true]{\compPhd profiling/\file}
\end{minipage}
\end{frame}


\subsection{Regular expressions}
\begin{frame}[fragile]
\frametitle{Regular expressions}
Versatile way to search and replace text using patterns.
\begin{itemize}
\item Horrible to master but tremendously powerful,
\item Understand them before the day you need them,
\item Bash has more tools (cat, grep, awk, sed, tr\dots),
\item Matlab has a few functions: \verb+strfind, regexp, regexprep+.
\end{itemize}
\end{frame}


\begin{frame}[fragile]\frametitle{Conditions}\centering
\begin{block}{if... elseif... else... end}
Use if you have conditions that are apparently not related.
\end{block}
\begin{block}{switch... case ... otherwise... end}
Use for matching among some values (strings or numerical).
\end{block}
\begin{exampleblock}{Example switch}\vspace{-5mm}
\begin{verbatim}swich class(f)
   case 'cell', disp('This is a cell');
   case 'double', disp('This is an array');
   case {'you', 'yourmom'}, disp('is so fat');
   otherwise, error('Not cool, man');
end
\end{verbatim}
\end{exampleblock}
\end{frame}


\begin{frame}[fragile]\frametitle{Loops}\centering
\begin{block}{for... end}
Use when you have an index that you control. \\
Always allocate memory by defining variables before the loop.
\end{block}
\begin{block}{while... end}
When \verb+for+ is not an option.
\end{block}
\begin{block}{break}
To terminate execution of a loop.
\end{block}
\begin{block}{continue}
To pass control to next iteration in a loop.
\end{block}
\begin{block}{return}
To exit a function when its job is done.
\end{block}
\end{frame}


\begin{frame}[fragile]
\frametitle{Regular expressions}\centering
\begin{exampleblock}{Example in Matlab}
\begin{verbatim}
labtext = importdata(labTXT);

labtext1 = strrep(labtext{1}, '.', ''); 

labtext2 = regexprep(labtext1, '\d\s*', ''); 

splitted = regexp(labtext2,'\s+', 'split');

upsplt = upper(sprintf('%s\n',splitted{:}));

upsped = regexprep(upsplt,'[0-9]|?|,|\.|!|,|;|-|:', '');
\end{verbatim}
\end{exampleblock}
\end{frame}


\begin{frame}[fragile]\frametitle{Regular expressions}\centering
\begin{exampleblock}{Example in Matlab}
\begin{verbatim}
labtext = importdata(labTXT);
% Get rid of dots, for consistency.
labtext1 = strrep(labtext{1}, '.', ''); 
% Get rid of initial numbers
labtext2 = regexprep(labtext1, '\d\s*', ''); 
% To write label in words.mlf; one word per line
splitted = regexp(labtext2,'\s+', 'split');
% uppercase
upsplt = upper(sprintf('%s\n',splitted{:}));
% Keep only letters
upsped = regexprep(upsplt,'[0-9]|?|,|\.|!|,|;|-|:', '');
\end{verbatim}
\end{exampleblock}
\end{frame}


\begin{frame}[fragile]\frametitle{Regular expressions}\centering
\begin{exampleblock}{Example in Bash}
\begin{verbatim}
#
pdftotext Thesis.pdf - | grep -o "\[?" | wc -l
\end{verbatim}
\end{exampleblock}
\end{frame}


\begin{frame}[fragile]\frametitle{Regular expressions}\centering
\begin{exampleblock}{Example in Bash}
\begin{verbatim}
# Approx. count number of missing citations in Thesis.pdf
pdftotext Thesis.pdf - | grep -o "\[?" | wc -l
\end{verbatim}
\end{exampleblock}
\end{frame}


\subsection{Interfacing with Matlab}
\begin{frame}{Interfacing with Matlab}
``Controlling complexity is the essence of computer programming'' \\
\hfill Brian Kernighan\\[5mm]
\pause
$\to$ If you can stick with one language only, do it!\\[5mm]
\pause
Reasons to use other languages within Matlab:
\pause
\begin{itemize}[<+(0)->]
\item Reduce runtime \\(`Oh, this loop would be much faster in C...')
\item Use existing packages \\(`Oh, if only there was a way to use this .jar?')
\item Reduce development time \\(`Oh, I know how to do this in Bash!')
\end{itemize}
\end{frame}


\begin{frame}{Interfacing with Matlab: C with mex-files}
\begin{itemize}
\item Very hard to debug (Matlab will crash often),
\item Requires specific commands, 
\item Might boost your code,
\item Takes significant development time.
\end{itemize}
\end{frame}


\begin{frame}{Interfacing with Matlab: C with mex-files}
\def\file{interfacing/C_mex}
\noindent\hspace{-7mm}\includegraphics[height=7.7cm, trim=0cm 0cm 0cm 2mm, clip=true]{\compPhd \file}
\end{frame}


\begin{frame}{Interfacing with Matlab: Java with jar-files}
\begin{minipage}{6cm}
\def\file{interfacing/java}
\begin{center}\includegraphics[height=7.7cm, trim=0cm 0cm 0cm 2mm, clip=true]{\compPhd \file}\end{center}
\end{minipage}%
\begin{minipage}{6cm}
\begin{itemize}
\item Need to know exactly what methods/classes you need,
\item Hard to debug, 
\item Juggling between Java and Matlab objects,
\item Will develop your insight of Matlab's inner working.
\end{itemize}
\end{minipage}
\end{frame}


\begin{frame}{Interfacing with Matlab: Bash with !}
On a Linux machine, you can execute bash commands using `!'\\
Ideal for some quick-n-dirty tricks
\def\file{interfacing/bash}
\begin{center}\includegraphics[height=7.5cm, trim=0cm 0cm 0cm 2mm, clip=true]{\compPhd \file}\end{center}
\end{frame}


\begin{frame}{Interfacing with Matlab: \LaTeX with fprintf...}
\def\file{interfacing/latex}
\begin{center}\includegraphics[width=11cm, trim=0cm 0cm 18cm 0mm, clip=true]{\compPhd \file}\end{center}
\end{frame}


\subsection{Programming Paradigms}
\begin{frame}{Programming Paradigms (Wikipedia)}\centering
\def\file{progparadigm/1179px-Programming_paradigms.svg}
\includegraphics[height=8.0cm]{\compPhd \file}
\end{frame}


\begin{frame}{Programming Paradigms in Matlab}
Matlab is (weakly) multi-paradigm:
\begin{itemize}
\item \textbf{Imperative Programming}: 
	Change program's state with commands describing how.
\item \textbf{Procedural programming}: 
	Procedures, also known as routines, subroutines, or functions contain a series of computational steps to be carried out.
\item \textbf{Object-Oriented Programming}: 
	Based on \textit{objects} and their \textit{methods and attributes} that interact with each other.
\end{itemize}
\end{frame}


\begin{frame}{Programming Paradigms in Matlab}
Complex matter:\\
\noindent[...] 
Matlab pretends to be multi-paradigm. It is optimized for imperative and procedural strongly typed operations. 
It supports OO, and extends to C/Cpp and Java. Contrary to popular opinion, Matlab is not strictly an interpreted language and 
does not always use JIT compilation. It uses an opaque optimization scheme 
[...]
\url{https://www.quora.com/What-programming-paradigm-does-MATLAB-follow}
\end{frame}

\begin{frame}{Object-Oriented Programming in Matlab}
Great if you work on big projects/with other people!\\
\url{https://uk.mathworks.com/company/newsletters/articles/introduction-to-object-oriented-programming-in-matlab.html}
\end{frame}


\subsection{Add-On Exploration}
\begin{frame}{Add-On Exploration}
\begin{block}{Add-Ons > Get Add-Ons}
Search and install `Megan Simulator'
\end{block}
\pause
\begin{block}{Simpler and faster than doing through your browser:}
\begin{itemize}
\item[\cg] Installed in the right place,
\item[\cg] Easy to manage, 
\item[\cg] Directly added to your path,
\item[\cg] All gathered in one place,
\item[\cg] You can write your own.
\end{itemize}
\end{block}
\end{frame}


%\subsection{Commenting}
%\begin{frame}{Commenting}
%\end{frame}


\subsection{Speeding-up}
\begin{frame}{Boosting Matlab Performance}
Parallelisation instead of  for loops\\
Logical operator\\
Preallocation\\
Master COW (Copy On Write)\\
Parallel processing toolbox\\
Removing bottlenecks after profiling\\
\dots\\
\url{https://uk.mathworks.com/help/matlab/matlab_prog/techniques-for-improving-performance.html}
\end{frame}


\subsection{Testing}
\begin{frame}{Testing}
If performance is a constraint/limitation, use a proper Testing framework.\\
Matlab offers a series of functions/classes/scripts to measure the performance of your program\\[3mm]
Special cases should become test cases.\\
{\tiny{\url{http://uk.mathworks.com/help/matlab/performance-testing-framework.html}}}
\end{frame}


\begin{frame}{Testing}\centering
\def\file{testing/runperf}
\vspace{-5mm}\hspace{-7mm}\includegraphics[height=7.5cm, trim=0cm 0cm 0cm 2mm, clip=true]{\compPhd \file}
\end{frame}


%\subsection{Version Control}
%\begin{frame}{Version Control}\centering
%\end{frame}


\subsection{Literate Computing}
\begin{frame}{Literate Computing}\centering
Interactive document that combines MATLAB code with embedded output, 
formatted text, equations, and images in a single environment called the Live Editor.\\[2mm]
\tiny{\url{https://uk.mathworks.com/help/matlab/matlab_prog/what-is-a-live-script.html}}
\end{frame}


\subsection{Unveiling Matlab Secrets}
\begin{frame}{Unveiling Matlab Secrets}
Yair Altman: \\
- Book `Accelerating MATLAB Performance'\\
- Awesome website \url{http://undocumentedmatlab.com/}\\[3mm]
Check help/doc of all functions you use frequently.\\[3mm]
Look at people's code.\\[3mm]
Google and fiddle
\end{frame}


%%%%%%%%%%%%%%%%%%%%
\section[SE]{Software Engineering}
%%%%%%%%%%%%%%%%%%%%


\begin{frame}{Software Engineering}
\pause
`The application of a systematic, disciplined, quantifiable approach to the development, operation, \\and maintenance of software.'\\
\hfill{\small{IEEE Standard Glossary of Software Engineering Terminology}}\\[6mm]
\pause
Programming: Producing code that works correctly.\\[3mm]
Software Engineering additionally:
\begin{itemize}
\item Readability, Maintainability, Documentation 
\item Unit Testing
\item Design
\item Code management, Task automation
\end{itemize}
\end{frame}


\subsection{Readability}
\begin{frame}{Readability}\centering
\textcolor{blue}{Write programs for people, not computers.}\\[3mm]
\begin{itemize}
\item Case sensitive names (weight1 rather than param1),
\item  Comment (whenever code is not obvious),
\item Give MWE (Minimal Working Example),
\item Do not recode (use available software),
\item Do not recode (small functions rather than copy-paste),
\item Write short functions that do one thing.
\end{itemize}
\end{frame}


\subsection{Automation}
\begin{frame}[fragile]\frametitle{Automation}\centering
\begin{itemize}
\item Use scripts (Matlab or bash) to execute code,
\item Keep logs 
\end{itemize}
\begin{exampleblock}{Example: Bash $\to$ Matlab $\to$ log}
\verb+matlab -nodisplay -nosplash < myscript.m > mylog.txt+
\end{exampleblock}
\end{frame}


\begin{frame}\frametitle{Automation}\centering
\def\file{xkcd/automation}
\includegraphics[height=6cm]{\compPhd \file}
\end{frame}


\subsection{Agile}
\begin{frame}{Agile Development}\centering
\begin{itemize}
\item Work in small increments,
\item Use version control where you save everything,
\item Measure your progress, 
\item Use it to rethink how you can improve your own efficiency.
\end{itemize}
\end{frame}


\begin{frame}{Agile Development}\centering
\def\file{xkcd/the_general_problem}
\includegraphics[height=4cm]{\compPhd \file}\\[3mm]
\pause
\textit{Make it work. Make it right. Make it fast.}
\end{frame}


\subsection{Open}
\begin{frame}{Open: Source, Code, Data}
\textbf{Share your code}, for ex. on GitHub; choose your \textbf{licenses}.\\[3mm]
Increasingly important: we start to be evaluated by our programming/SE skills.\\[3mm]
Institutions (universities and governements) are being pushed to share their code and data.\\[3mm]
\pause
\begin{center}\textcolor{blue}{A code you can't share is a code that is wrong.}\end{center}
\pause \hfill Alban, 12th of October 2016, 2:18am
\end{frame}


\subsection{Summary of Best Practices}
\begin{frame}{Summary of Best Practices}\centering
\begin{tabular}{c c}
Readability & Automation\\
Agile Programming & DRY Principle\\
Correctness & Optimisation \\
Documentation & Collaboration
\end{tabular}
\end{frame} 


\subsection{Tools within Notts}
\begin{frame}[fragile]\frametitle{Resources specific to PhDs Nottingham}
High-Performance Computing facility (HPC),\\
Maths servers (Shrek, Pegasus,...), \\
Nottingham  PhD students,\\
Organise you own seminar :)
\end{frame}


\subsection{Software Engineering Resources}
\begin{frame}{Software Engineering Resources}
Greg Wilson\\
\small{\url{http://software-carpentry.org/lessons/}}\\[2mm]
Best Practices for Scientific Computing, PLoS Biology 2014. \\
\small{\url{http://journals.plos.org/plosbiology/article?id=10.1371/journal.pbio.1001745}}\\[2mm]
Get some apps for your daily learning\\
{\tiny{\url{https://medium.com/the-mission/9-places-to-learn-how-to-code-in-15-minutes-or-less-a-day-7eb730e4fc82\#.7gxej497p}}}
\end{frame}


\subsection{Bash One-Liner}
\begin{frame}[fragile]\frametitle{Favourite Bash One-Liner from this PhD}
To connect as \verb+pmxlol+ to the \verb+pegasus+ server, \\
 reach the existing and detached \verb+screen+ session \verb+myscreen+,\\
 launch Matlab within it, \\
 execute the script \verb+myscript.m+, \\
 save output in a log called \verb+mylog.txt+:\\
\pause
\begin{verbatim}ssh pmxlol@pegasus.maths.nottingham.ac.uk \
 "screen -S myscreen -X stuff \
 \"matlab -nodisplay -nosplash < myscript.m > mylog.txt \
 $(echo -ne '\r')\""\end{verbatim}
 \pause
{\tiny{Links to explain the different bits:\\
\url{https://code.tutsplus.com/tutorials/ssh-what-and-how--net-25138}\\
\url{https://www.gnu.org/software/screen/manual/screen.html}\\
\url{http://unix.stackexchange.com/questions/13953/sending-text-input-to-a-detached-screen}\\
\url{http://stackoverflow.com/questions/33187141/how-to-call-matlab-script-from-command-line}\\[-2.25mm]
\url{http://www.tldp.org/ldp/abs/html/io-redirection.html}}} 
\end{frame}


\begin{frame}
\textbf{Master Foo and the Recruiter}\\[4mm]
\footnotesize{
A technical recruiter, having discovered that that the ways of Unix hackers were strange to him, 
sought an audience with Master Foo to learn more about the Way. Master Foo met the recruiter in the HR offices of a large firm.\\[2mm]
The recruiter said, "I have observed that Unix hackers scowl or become annoyed when I 
ask them how many years of experience they have in a new programming language. Why is this so?"\\[2mm]
Master Foo stood, and began to pace across the office floor. The recruiter was puzzled, and 
asked "What are you doing?"\\[2mm]
"I am learning to walk," replied Master Foo.\\[2mm]
"I saw you walk through that door" the recruiter exclaimed, "and you are not stumbling over your 
own feet. Obviously you already know how to walk."\\[2mm]
"Yes, but this floor is new to me." replied Master Foo.\\[2mm]
Upon hearing this, the recruiter was enlightened.
}
\end{frame}


\subsection{References}
\begin{frame}{References}
Randall Munroe's \\\url{https://xkcd.com/}\\[2mm]
Eric Steven Raymond's \\\url{http://catb.org/esr/writings/unix-koans/index.html}\\[2mm]
Unix Philosophy \\\url{http://www.catb.org/esr/writings/taoup/html/index.html}\\[2mm]
Mario Mulansky's `Programming for Scientists'\\[2mm]
Best Practices for Scientific Computing, PLoS Biology 2014. 
\url{http://journals.plos.org/plosbiology/article?id=10.1371/journal.pbio.1001745}
\end{frame}


%%%%%%%%%%%%%%%%%
\section[T]{}
%%%%%%%%%%%%%%%%%

\subsection{Thanks}
\begin{frame}{Thanks for your attention!}\centering
Merci pour votre attention !
\end{frame}


\begin{frame}{To meditate on...}\centering
\includegraphics[width=14.5cm, trim=0cm 3.5cm 0cm 0cm, clip=true]{\compPhdGeneral UnixPhilosophy2.png}
\end{frame}


\begin{frame}
\textbf{Master Foo and the Ten Thousand Lines}\\[4mm]
\tiny{
Master Foo once said to a visiting programmer: "There is more Unix-nature in one l
ine of shell script than there is in ten thousand lines of C."\\[2mm]
The programmer, who was very proud of his mastery of C, said: "How can this be? 
C is the language in which the very kernel of Unix is implemented!"\\[2mm]
Master Foo replied: "That is so. Nevertheless, there is more Unix-nature in one 
line of shell script than there is in ten thousand lines of C."\\[2mm]
The programmer grew distressed. "But through the C language we experience the 
enlightenment of the Patriarch Ritchie! We become as one with the operating 
system and the machine, reaping matchless performance!"\\[2mm]
Master Foo replied: "All that you say is true. But there is still more Unix-nature 
in one line of shell script than there is in ten thousand lines of C."\\[2mm]
The programmer scoffed at Master Foo and rose to depart. But Master Foo 
nodded to his student Nubi, who wrote a line of shell script on a nearby whiteboard, 
and said: "Master programmer, consider this pipeline. Implemented in pure C, would it not span ten thousand lines?"\\[2mm]
The programmer muttered through his beard, contemplating what Nubi had written. Finally he agreed that it was so.\\[2mm]
"And how many hours would you require to implement and debug that C program?" asked Nubi.\\[2mm]
"Many," admitted the visiting programmer. "But only a fool would spend the time to do 
that when so many more worthy tasks await him."\\[2mm]
"And who better understands the Unix-nature?" Master Foo asked. "Is it he who writes 
the ten thousand lines, or he who, perceiving the emptiness of the task, gains merit by not coding?"\\[2mm]
Upon hearing this, the programmer was enlightened.
}
\end{frame}

\end{document}
